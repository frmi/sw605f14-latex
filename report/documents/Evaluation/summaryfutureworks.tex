We found several candidates for improvements of \launcher.
In this section, we only describe the three we deem to have the highest priority.
An exhaustive list of improvements can be found in \cref{appendix:futureworks}.

\paragraph{Copying settings from one user to another} has not been accomplished due to time constraints.
The clients requested the possibility to copy settings from one user to another, this is described in \cref{sec:sprint2:secondmeeting}.
Since this is an actual requirement, we strongly recommend this to be implemented as soon as possible.

\paragraph{Downloading pictograms} while the tablet is being used, is a feature which we find very important.
Currently, when \launcher is started after a system restart, the synchronisation of data with the remote database is executed.
This synchronisation can last up to 30 minutes, during which the user has to wait.
It would be ideal to only download the most essential data, such as profiles and their relations, and thereafter give the user the possibility to login.
\launcher could then continue downloading other data in the background while the tablet is being used.

\paragraph{The loading of applications} into views is suggested to be reimplemented. 
The layout in which applications are being loaded, is currently a linear layout where another linear layout is added acting as a row.
This results in multiple calculations of padding around each application, depending on the chosen icon size.

Instead, the mechanism should be implemented using a grid view.
This makes it possible to add the applications using an adapter and the spacing around the applications would be set automatically.

This increases performance as described in \cref{sec:settingslistadapter}, and would simplify the algorithm used to add the applications to a view significantly.