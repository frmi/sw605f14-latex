Half way through the first sprint, semester coordinator Ulrik Nyman, requested that at the end of the semester, the applications should be available on Google Play.
This request made all groups in the multi-project realise, that it would not be favourable to have the local database distributed as a stand-alone application, since the \giraf system as a whole is dependant on the data stored in the local database.
The reason for integrating the local database into the \launcher project, is that most of the other applications in the \giraf suite depend on data from the launcher.
In this way, they are required to be started from \launcher.

\Cref{sec:developments:localdbtolauncher} clarifies the technical aspect of the integration.