Google Analytics is a tool for gathering statistics on the use of mobile applications and websites. 
It is developed to help marketers reach their target groups, by analysing how their applications are used.

Most of the possibilites of Google Analytics are not directly relevant to the \giraf project, however it has a feature which could be of great use.
Google Analytics can automatically record uncaught exceptions that occur on the users' devices.
This is a useful feature for multiple reasons. 
The users may not report an application crash to the developers, which prevents the them from addressing the underlying issue. 
Furthermore, Google Analytics continuously monitors problems with the applications, even during the eight months each year, where no students are assigned to the project. 
When a new team of students resume the work, they can immediately start addressing issues that may have shown up during this period.

A member of this group initially came up with the idea of integrating Google Analytics into the \giraf project.
He was made responsible for instructing groups in setting up the tool and make sure that it was implemented correctly.
He also wrote a tutorial, which was distributed to the other groups.