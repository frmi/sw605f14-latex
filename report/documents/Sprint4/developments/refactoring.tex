\subsection{Splitting up LauncherUtility}\label{sect:sprint4:refactoring}

A quick analysis of the code reveals that \lstinline|LauncherUtility| is now too big, spanning over 900 lines with static methods for many different purposes.
By dividing the class into smaller classes, each with a certain purpose, the maintainability, readability, and understandability of the code is increased.
Furthermore, many of the existing functions are either simplified or removed due to lack of use or redundancy.

\lstinline|LauncherUtility| is split into three different classes:

\begin{itemize}
\item \lstinline|LauncherUtility| remains as a class.
It now hosts many of the minor functions, that do not warrant their own class, but is used in a variety of activities.
Examples are methods getting helper functions from \textit{OasisLib}, information about the current user, session expiration methods, and handling debug mode.
The latter is described in \cref{appendix:debugmode}.
\item \lstinline|ApplicationControlUtility| is created.
This class contains methods that are used to either return different lists of applications or check if a certain list lives up to a given criteria.
These methods are used when sorting which applications should be marked in \settingsactivity or shown in \homeactivity.
\item \lstinline|AppViewCreationUtility| is created as well.
This class is used for the creation of views that contain an application.
The views are of the class \lstinline|AppImageView|, which extends the image view class and has functionality used in \settingsactivity, when selecting applications. 
This class has state field, which indicates if the application is selected or not. 
It also implements functionality to toggle this state, between selected or not selected, changing the background of the view to indicate that it has been selected.
\end{itemize}

The two new classes have their own specific purposes, while \lstinline|LauncherUtility| handles a collection of smaller purposes.

A discussion on how this could have been improved even further, can be read in \cref{sec:eval:discussion}.