\subsection{Disabling the Drawer}
As described in \cref{sec:sprint2:firstmeeting,sec:sprint2:secondmeeting}, the purpose of the drawer is founded in a misinterpretation.
It was thought by the previous groups working with \launcher as being a requirement to be able to colour each of the \giraf applications.
However, at the client meetings, it was clarified that no such requirements for the applications exist.
What the clients want is to be able to switch the colours of \textit{pictograms}, being able to also show them in black and white.
This is due to some citizens handling colours better than black and white and vice versa, as referred to in \cref{sec:sprint2:clarification}.\\

Some additional problems are found with the drawer.
These are mostly centred upon other applications absorbing the colour given to them by \launcher, and using that as their main theme.
However, this means that for each possible colour \launcher can give an application, it needs to implement a colour scheme that fits with the given colour.
This is especially a problem for applications using a wide variety of colours.\\

Since the colouring functionality of the drawer is not a requirement per se and caused problems in other applications, it is decided to disable the drawer, as it now serves no particular purpose.
However, the code for animating the drawer is preserved, in case the functionality at a later time will prove useful.
This way, should a future group need to implement the drawer, they merely have to uncomment parts of the code.