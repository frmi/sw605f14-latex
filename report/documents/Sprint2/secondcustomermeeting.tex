\section{Second Client Meeting}\label{sec:sprint2:secondmeeting}
The meeting is held the 3rd of April 2014 with two representatives Mette Als Andreasen and Kristine Niss Henriksen from the nursery for children with ASD, called Birken.
A summary of the meeting is found in \cref{appendix:secondmeeting}.
The presented prototypes are the same as those in the previous meeting (\cref{fig:prototypes}).

\subsection{Feedback}
This meeting was conducted in the same manner as \Cref{sec:sprint2:firstmeeting}, therefore please refer to the Feedback section from this meeting in \Cref{sec:firstmeeting:feedback}.

\subsubsection{Profile Selector}
The profile selection tool is preferred as a combination of the two options -- as a selector inside each launched application and as a tool-tip from clicking the profile picture in \launcher.

\subsubsection{Settings}
The settings tool is preferred also as a combination.
Accessing the settings related to each application from inside that particular application is the most important of the two. 
Accessing the settings is achieved by pressing a button with an image of a cogwheel.
Furthermore, pressing the settings button while in \launcher should open an activity, where settings for both \launcher itself and all installed applications can be manipulated.
An important aspect of the settings is the ability to copy settings from one user to another.

\subsubsection{External Applications}
Lastly, the idea of being able to add applications from outside the \giraf suite through \launcher, including downloading new applications from Google Play, is well received.
The clients especially like the idea of enabling applications on a `per user'-basis.
This functionality is advised to be in the settings activity found in \launcher.

\subsubsection{Android Buttons}
Apart form the prototype, the solution of overriding the \textit{Home}, \textit{Back} and \textit{Multitasking} buttons is accepted, but the clients prefer it to be disabled completely.
It is suggested by the clients to utilize the Android equivalent of iOS' ``Restricted Access'', to achieve the preferred result.
The research is described in \cref{sec:sprint2:backlog}.

\subsection{Clarification of Issues}\label{sec:sprint2:clarification}
One major issue regarding conflicting requirements is clarified at this meeting.
Previously, there was a requirement of a basic black-and-white colour scheme, with options to change scheme.
It is then clarified that this requirement is related only to the pictograms and not to the program as a whole, meaning the pictogram should in general be black and white and those in colour should be able to be converted to black and white.
The drawer component in \launcher, responsible of changing the colour of applications was not a requirement, and therefore is not particularly important to the clients.

\subsection{Communication}
Communication with these clients is more smooth than at the previous meeting, however, there are still comprehension barriers.
Most barriers are related to the clients not understanding what features would be trivial to implement and which would be exceedingly difficult.
Furthermore, questions asked by the group often have to be reformulated for the clients to understand them completely.
However, the clients often explain how each question is understood, and always point out elements they do not understand.
This helps greatly in case of communication problems.

These clients furthermore express more concretely what is desired from the application.
Additional questions from the group are still required, but are answered to completion.

Overall, the meeting is regarded as a success, with constructive feedback.