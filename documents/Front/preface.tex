\newcommand{\headerPreface}{Preface}
\cleardoublepage
\phantomsection
\pdfbookmark{\headerPreface}{chap:preface}
\chapter*{\headerPreface}\label{chap:preface}

\style[inline]{''Multi-project'' should be written as it is written here.}

\style[inline]{Write method names as \lstinline!fooBar()!}

\style[inline]{Distinguish between activities as concepts and activities as classes. }

\style[inline]{Activity (concept): capital initials (Home Activity).}

\style[inline]{Present tense.}

\style[inline]{Application names in italics: \textit{Pictooplæser}}

\style[inline]{Terminology: children $\rightarrow$ citizens (not in code). User, guardian. Costumers $\rightarrow$ clients.}

\style[inline]{Proper capitalization.}

\style[inline]{Correct quote types.}

\style[inline]{Maybe write introductions to sprints in Sprint Planning?}

\thilemann[inline]{Adjust all intros and summaries according to content.}

\jesper[inline]{Describe new specifications, and refer to old ones.}

\style{Citations right before or after periods? Have to be consistent!}

This report is created by Software Engineering students as a Bachelor project at Aalborg University, spring semester 2014.
To read and understand the report, it is expected that the reader has a background in Computer Science in the light of the technical contents.
The Android APIs are not described directly in this report and thus the reader is encouraged to explore the introduction to Android given by \citet{androidIntroduction}.

The project is organized in multiple groups, where 16 bachelor groups have collaborated to create an Android system for autistic citizens and their guardians.
The multi-project has been in progress since 2011, and is a collaboration between Aalborg Municipality, Aalborg University and several institutions working with autistic citizens.
The multi project is further described in \cref{chap:giraf}.

\thilemann{Should we cite multi-project guidelines when mentioning them the first time?}
Since multiple project groups are collaborating towards developing a complete system, it is specified by the multi-project guidelines that the groups must work in four sprints.
The structure of this report is divided into chapters that logically follows from these guidelines.
Each sprint chapter is then further subdivided into sections that are similar for each sprint.
A section can for example be ``Analysis and Design'' or ``Developments''.
A particular section is only added if its describing activity has been present during the specific sprint.\\

It is important to note, that the applications in the multi-project were renamed during this semester, as a decision taken in \textbf{INSERT HERE}.\thilemann{Where was this decided?}
As a result, many applications have been given Danish names, which will not be translated to English in this report.

It should also be noted that some developments are not described by analysis nor design; this follows from the fact that many developments are a consequence of having tasks in backlog from sprint to sprint, and that these often is the effect of refactoring existing code.

References and citations are given with number notation, for example as: \citet{launcher2011}, or without listing the authors as: \cite{launcher2011}. 

We would like to thank our lecturers and our supervisor, Hua Lu, for excellent cooperation during the project work.
Furthermore, we would like to thank the other project groups for the educational collaboration performed during the project.
\jesper[inline]{Fix this up in the end.}
\thilemann[inline]{Think its good :)}
