\newcommand{\headerPreface}{Preface}
\cleardoublepage
\phantomsection
\pdfbookmark{\headerPreface}{chap:preface}
\chapter*{\headerPreface}\label{chap:preface}

\style[inline]{''Multi-project'' should be written as it is written here.}

\style[inline]{Write method names as \lstinline!fooBar()!}

\style[inline]{Distinguish between activities as concepts and activities as classes. }

\style[inline]{Activity (concept): capital initials (Home Activity).}

\style[inline]{Present tense.}

\style[inline]{Application names in italics: \textit{Pictooplæser}}

\style[inline]{Terminology: children $\rightarrow$ citizens (not in code). User, guardian. Costumers $\rightarrow$ clients.}

\style[inline]{Proper capitalization.}

\style[inline]{Correct quote types.}

\style[inline]{Maybe write introductions to sprints in Sprint Planning?}

\thilemann[inline]{Adjust all intros and summaries according to content.}

\jesper[inline]{Describe new specifications, and refer to old ones.}

This report is created by Software Engineering students as a Bachelor project at Aalborg University, spring semester 2014.
To read and understand the report, it is expected that the reader has a background in Computer Science in the light of the technical contents.
The Android APIs are not described directly in this report and thus the reader is encouraged to explore the introduction to Android given by \citet{androidIntroduction}.

The project itself is a multi project, where 16 bachelor groups have collaborated to create an Android system for autistic citizens and their guardians.
The multi project has been in progress since 2011 and is a collaboration between Aalborg Municipality, Aalborg University and several institutions for autistic citizens.

The multi project as a whole is described in \cref{chap:giraf}.\\

Since multiple project groups are collaborating towards developing a complete system, it is specified by the multi-project guidelines, that the groups must work in four sprints.
The structure of this report is divided into chapters that logically follows from these guidelines.
Thus there is a chapter for each sprint, \cref{chap:sprint1} for the first, \cref{chap:sprint2} for the second, \cref{chap:sprint3} for the third and \cref{chap:sprint4} for the fourth and final sprint.
Each sprint chapter is further subdivided into:

\begin{description}
\item[Sprint Overview] \hfill \\
In this section we describe the overall goals and activities of the sprint.
\item[Analysis and Design] \hfill \\
In this section we describe the analytical processes and design decisions related to the work carried out during this sprint
\item[Developments] \hfill \\
In this section we describe how we solved the tasks of the sprint.
\item[Sprint Review] \hfill \\
In this section we evaluate the success of the sprint and detail a backlog for the coming sprint
\end{description}

The second and the fourth sprint conducted formal tests, which are described after the developments section of these sprints.
Following the chapters describing the four sprints, \cref{chap:collaboration} focuses on the work that is done in collaboration with other groups.
Lastly, the results of the project is presented in \cref{chap:evaluation}.

It is important to note, that the applications in the multi project were renamed during this semester.
As a result, many applications have been given Danish names which will not be translated in this report.

References and citations are given with number notation, for example as: \citet{launcher2011}, or without listing the authors as: \cite{launcher2011}. 

We would like to thank our lecturers and our supervisor, Hua Lu for excellent cooperation during the project work.
Furthermore, we would like to thank the other project groups for the educational collaboration performed during the project.
\jesper[inline]{Fix this up in the end.}
