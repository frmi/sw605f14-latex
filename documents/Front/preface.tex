\newcommand{\headerPreface}{Preface}
\cleardoublepage
\phantomsection
\pdfbookmark{\headerPreface}{chap:preface}
\chapter*{\headerPreface}\label{chap:preface}

\style[inline]{Write method names as \lstinline!fooBar()!}

\style[inline]{Distinguish between activities as concepts and activities as classes.}

\style[inline]{Present tense.}

\thilemann[inline]{Adjust all intros and summaries according to content.}

\jesper[inline]{Describe new specifications, and refer to old ones.}

This report is created by Software Engineering students as a Bachelor project at Aalborg University, spring semester 2014.
To read and understand the report, it is expected that the reader has a background in Computer Science in the light of the technical contents.
The Android APIs are not introduced in this report and thus the reader is encouraged to explore the introduction to Android given by \citet{androidIntroduction}.

Since multiple project groups are collaborating towards developing a complete system, the structure of this report is divided into chapters that logically follows from working in sprints specified by the multi-project guidelines.
Thus there is a chapter for each sprint, which is further subdivided into:

\begin{description}
\item[Sprint Overview] \hfill \\
In this section we describe the overall goals and activities of the sprint.
\item[Analysis and Design] \hfill \\
In this section we describe the analytical processes and design decisions related to the topics 
\item[Developments] \hfill \\
In this section we describe how we solved the tasks of the sprint.
\item[Sprint Review] \hfill \\
In this section we discuss lessons learned during the sprint, including collaboration problems, and erroneous use of development methods.
\end{description}

Following the chapters describing the four sprints, \cref{chap:collaboration} focuses on the work that is done in collaboration with other groups.
Lastly, the results of the project is presented in \cref{chap:evaluation}.

References and citations are given with number notation, for example as: \citet{launcher2011}, or without listing the authors as: \cite{launcher2011}. 

When writing ``we", it is referring to the members of the project group.

We would like to thank our lecturers and our supervisor, Hua Lu for excellent cooperation during the project work.

\vagner[inline]{Add remark on changed application names.}
\jesper[inline]{Fix this up in the end.}
