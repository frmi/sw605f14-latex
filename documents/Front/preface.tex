\newcommand{\headerPreface}{Preface}
\cleardoublepage
\phantomsection
\pdfbookmark{\headerPreface}{chap:preface}
\chapter*{\headerPreface}\label{chap:preface}

\style[inline]{Citations right before or after periods? Have to be consistent!}

\style[inline]{Make sure to be more concise in our formulas.}

\style[inline]{Scrum master $\rightarrow$ Scrum Master.}

This report is created by Software Engineering students as a Bachelor project at Aalborg University, spring semester 2014.
To read and understand the report, it is expected that the reader has a background in Computer Science in the light of the technical contents.
The Android APIs are not described directly in this report and thus the reader is encouraged to explore the introduction to Android given by \citet{androidIntroduction}.

The project is organized in multiple groups, where 16 bachelor groups have collaborated to create an Android system for autistic citizens and their guardians.
The multi-project has been in progress since 2011, and is a collaboration between Aalborg Municipality, Aalborg University and several institutions working with autistic citizens.
The multi-project is further described in \cref{chap:giraf}.

Since multiple project groups are collaborating towards developing a complete system, it is necessary to have a common work process.
Therefore, it was decided by the semester coordinator to have the groups work in four sprints.
The structure of this report is divided into chapters that logically follows from this work process.
Each sprint chapter is then further subdivided into sections that are similar for each sprint.
A particular section is only added if it describes activities carried out during the specific sprint.\\

It is important to note, that the applications in the multi-project were renamed during this semester.
The new names better describe the essence of the applications.
As a result, many applications have been given Danish names, which will not be translated to English in this report.

It should also be noted that some developments are not described by analysis nor design.
This follows from the fact, that many developments are a consequence of having tasks in the backlog from a preceding sprint.
Furthermore, these developments are often centered around the refactoring of existing code.

References and citations are given with number notation, for example as: \citet{launcher2011}, or without listing the authors as: \cite{launcher2011}.

We would like to thank our lecturers and especially our supervisor, Hua Lu, for excellent cooperation during the project work.
Furthermore, we would like to thank the other project groups for the educational collaboration performed during the project.