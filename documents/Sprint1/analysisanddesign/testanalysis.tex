\section{Functional Tests}\label{sec:sprint1:testing}

As one of our goals is to increase the reliability of \launcher, we decided to conduct functional testing on the application.
We studied the specifications formulated by the earlier development teams in \citet{launcher2011,launcher2012}, and then studied how well \launcher correspond to these specifications, by using the program in a number of different ways. 

Furthermore, a number of test cases are not directly derived from the old specifications, but from what we find reasonable to expect of the application, based on our own knowledge of the system.

The tests described here are performed by operating the running application, and are explicitly not based on our knowledge of the source code. 
This characterizes our tests as being dynamic black box testing. 

A number of test cases that would have been relevant, is not performed in this sprint as they would have been too time consuming. \frederik{Skal vi skrive hvilke dette eventuelt kunne dække over?}
The tests require significant changes to the test data built into the \giraf system, or repeatedly installing and un-installing applications from the device.
As there, at the time, was no central repository of stable application builds, most of the time would be spent building, installing and resetting different applications.

We only describe the results we find the most interesting in this section; mainly tests that failed in a significant way.

\subsubsection*{Specifications}
In the 2011 report by \citet{launcher2011}, the following specification for \launcher is presented:

\begin{quote}
\begin{itemize}
	\item \textit{List only applications that is part of the \giraf system.}
	\item \textit{List only applications that the user is able to use, according to the user's capabilities.}
	\item \textit{List only applications if their usage is not limited by the current location or time.}
	\item \textit{Only display application names if the user is able to read.}
\end{itemize}
\end{quote}

Common to all these, is that they are either difficult to test, or simply not implemented (some may have been to some degree in 2011, but then removed by the 2012 team).
The difficulty stems from the necessity of manipulating test data, that is hard coded into the local database of each device. 
We decided it would be more reasonable to revisit these test cases when \giraf's database system is ready for use.

The 2012 report by \citet{launcher2012} contains no explicit specifications, but does contain a list of use cases. 
These form the basis of our tests:

\begin{quote}
\begin{itemize}
	\item \textit{New guardian log in.}
	\item \textit{Configuring an application for a child.}
	\item \textit{Launching an application for a child in guardian mode.}
	\item \textit{Letting a child use an application for a limited time.}
\end{itemize}
\end{quote}

Note that their report also contains an explanation of each use case. 
We leave these out for brevity.
