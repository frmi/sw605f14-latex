\subsection{The Error in Authentication Activity}
When starting \launcher for the first time, the application crashed with an \lstinline{OutOfMemoryError} upon reaching the \authenticationactivity.
The problem is caused by an animation in the activity, implemented as an instance of the Android class \lstinline{AnimationDrawable}, a class designed to create animations from a sequence of images.
According to the Android documentation \citet{androidreference}, the implementation is technically correct, but it is found that when an instance of \lstinline{AnimationDrawable} is created, it loads the entire sequence of images into main memory of the device at once.
The images are contained within the application in the compressed Portable Network Graphics (PNG) format.
However, as they are loaded, Android converts each image to a 32-bit bitmap, which is 4 bytes per pixel, making the images take up about 20 times the space.
The problem with the original implementation is, that the animation consists of 90 pictures, thus resulting in the animation using about 40 MB of memory when converted to bitmaps. 
We hypothesized that this was the source of the \lstinline{OutOfMemoryError}.

Our solution is to create a new class which simulates the behaviour of \lstinline{AnimationDrawable}, but only loads one frame at a time.
This is achieved by a recursive method, which changes the shown image automatically.
The image references are stored in a static array, which is iterated repeatedly using the method \lstinline|postDelayed()|, that runs a given method after a set delay.\\\\

In order to further ensure the reliability of \launcher, functional tests of the application are also carried out.
These are described in the following section.