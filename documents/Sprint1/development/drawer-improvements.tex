\subsection{Improved Behaviour of the Drawer}
As mentioned in \cref{sec:launcher:drawer}, the drawer is opened by pressing and dragging the vertical bar to the right.
The code working the animation of the drawer is based on an touch listener, using a motion event which is fired only by movement.
The event sets the position of the entire drawer, panel and bar to the exact point it has just been moved to before redrawing the elements.
Since the position is not limited, the user can leave it in any position between fully extended and closed state as explained in \cref{sec:drawer:behaviour}.

To change this behaviour .... \thilemann{Write!}

Opening and closing the improved drawer was implemented with an \lstinline{OnTouchListener} through the \lstinline{MotionEvent.ACTION_DOWN} event and a standard Android \lstinline{TranslateAnimation}.
Pressing the bar fires the event and begins the animation.
A \lstinline{boolean} variable determines whether the drawer was open or closed when the event fired and thus if the animation should translate left or right.
By also adding an \lstinline{OnDragListener} to the drawer, the animation would also play when dragging and dropping colours from the panel; closing the drawer on \lstinline{DragEvent.ACTION_DRAG_STARTED} and opening it again on \lstinline{DragEvent.ACTION_DRAG_ENDED}.

Having the \lstinline{ScrollView} containing the application icons remain static during the animation, was solved by reorganising the source code responsible for the user interface layout. 
Previously, all elements were declared in the XML layout file, while positioning of the elements was set to static values in \lstinline{HomeActivity}.
This was refactored to have the all positioning declared dynamically depending on screen size, apart from the positioning of the \lstinline{ScrollView}.\thilemann{Stefan, write intelligent stuff about dynamic positioning of the ScrollView}