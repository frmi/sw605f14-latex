\subsection{Drawer Improvements}\label{sec:developments:drawerimprovements}
As mentioned in \cref{sec:launcher:drawer}, the drawer was opened by pressing and dragging the drawer border.
The code working this animation was based on an \lstinline{OnTouchListener}, focusing mainly on the \lstinline{MotionEvent.ACTION_MOVE} event.
\lstinline{MotionEvent.ACTION_MOVE} would set the position of the entire drawer, panel and bar, to the exact point it had currently been moved to and redraw the elements.
Because the \lstinline{ScrollView} containing the application icons was set to \lstinline{android:layout_toRightOf} in the layout XML files, the \lstinline{ScrollView} would adjust itself to the new position each time the event fired.

The panel of colours was implemented using a \lstinline{GridView} - using the command \lstinline{AppColors.setAdapter(new GColorAdapter(this));}, the \lstinline{GColourAdapter} from \lstinline{OasisLib} would create the correct panel.

Selecting a colour and assigning it to an application worked by means of an \lstinline{OnDragListener} called \lstinline{GAppDragger}.

\subsubsection{The Improved Drawer}

Opening and closing the improved drawer was implemented with an \lstinline{OnTouchListener} through the \lstinline{MotionEvent.ACTION_DOWN} event and a standard Android \lstinline{TranslateAnimation}.
Pressing the bar fires the event and begins the animation.
A \lstinline{boolean} variable determines whether the drawer was open or closed when the event fired and thus if the animation should translate left or right.
By also adding an \lstinline{OnDragListener} to the drawer, the animation would also play when dragging and dropping colours from the panel; closing the drawer on \lstinline{DragEvent.ACTION_DRAG_STARTED} and opening it again on \lstinline{DragEvent.ACTION_DRAG_ENDED}.

Having the \lstinline{ScrollView} containing the application icons remain static during the animation, was solved by reorganising the source code responsible for the user interface layout. 
Previously, all elements were merely declared in the XML layout file, while positioning of the elements was statically set in the \lstinline{HomeActivity} load code - over 100 lines of parameter setting code.\vagner{maybe not include mocking the previous group?}
This was refactored to have the all positioning declared dynamically depending on screen size, apart from the positioning of the \lstinline{ScrollView}.\vagner{Stefan, write intelligent stuff about dynamic positioning of the ScrollView}
\vagner{Add documentation about the icon indicating opening and closing of the drawer}

The improved drawer fulfils all desired the previously specified requirements, and provides a smoother and more pleasant experience for the user, while refactoring of the acting code provides clarity and readability of developers.\jesper{Is this a little arrogant, considering we haven't actually shown it to users?}\vagner{True, and in sprint 2 we learn that the user doesnt even need the drawer. However, this was what we thought was requirements at the time of writing.}

