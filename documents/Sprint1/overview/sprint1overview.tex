This section covers the layout of the first sprint of the project.
This sets apart from the following sprints, since it focuses on getting to know \giraf in details, setup of the development environment and getting the system up-and-running.

\subsection{Planning}\label{sec:sprint1:planning}
The agenda for the first sprint meeting is to get an overview of the development taken place during the preceding semesters and to get development started.
To begin with, each multi-group have chosen a \giraf application to work on, even though this selection is subject to change during the following sprints.
We have selected the \launcher project, which is further described in \cref{sec:launcher}.

Some activities are also planned to get to know Android development and the \giraf applications:

\begin{itemize}
\item Android-workshop\\
Introductory workshop about Android development that can be attended if one does not have any prior experience with Android.
\item Install party\\
Intended to get the \giraf development environment up-and-running as last years students worked on. 
This event failed due to the unstructured nature of the setup of the environment (Eclipse IDE and ANT build system) and thus not every student was able to get \giraf to work.
\end{itemize}

It is also decided that the following tools are used:

\begin{itemize}
\item \textbf{Redmine} (project management web application)
\item \textbf{Git} (distributed version control system)
\item \textbf{Jenkins} (continuous integration)
\item \textbf{Android Studio} (Android development environment)
\item \textbf{Gradle} (build automation control)
\end{itemize}

This selection differs from that of last year, since Git has replaced SubVersion (SVN), Android Studio has replaced Eclipse IDE and Gradle as replaced ANT.

\subsection{Objectives}\label{sec:sprint1:objectives}
Since this is the first sprint and we thus have no prior knowledge of \giraf, this sprint can not focus on any new developments.
Therefore, attention is given to the work done by last years students working on the project, and thus looking for improvements and fixing bugs.
During the first sprint meeting, it was clear based, on semester coordinator Ulrik Nyman, that we should strive to deliver a functioning suite of applications after the first sprint.
It should not be understood as a version to be delivered to the customers, but merely an installable version that can run without any crashes. 

A separate project group was tasked with handling all negotiations with the customers during this first sprint, in order to compile a backlog for the second sprint and specify a requirements specification for the entire project.

The rest of this chapter focuses on the \launcher.









