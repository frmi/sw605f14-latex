While the multi-project was forced to use the Scrum development process, the individual groups were free to chose their method.
We chose to adopt Scrum as well.
The morning meetings provided a great overview of the tasks to be carried out during the day.
Furthermore, the sprint backlog compiled after each sprint, left us with a good overview of tasks to be completed during each sprint.

However, we deviated from Scrum in several ways.

First of all, we did not use a Scrum Master in the group.
Since the group only consisted of four member, it was deemed to be unnecessary.

Second, though we did have sprint backlogs, they were not as detailed as they should be.
This also lead to many tasks appearing during sprints, which were not accounted for in the sprint backlog, and slowed down the group.

Lastly, the act of estimating the time each task would require was mostly omitted by the group.
While \citet{larmanAgile} points out that this is one of the more difficult parts of Scrum, the group rarely made even an attempt to do so.\\

Despite these deviations, it is the general consensus of the group, that the work process was a success.


The next section discusses further improvements that could be made to the code of \launcher.