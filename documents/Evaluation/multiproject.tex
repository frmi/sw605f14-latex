Reflecting on the multi-project work flow, a number of things could have been improved, and a few things should be retained by later teams. The most important will be described here.

\subsection{Use of Scrum}
\label{sec:eval:multiproject:scrum}
The use of Scrum on the multi-project level did not work out well. While Scrum, as described by \citet{scrumGuide}, requires that at least two roles are filled, the Scrum master and the product owner, only a Scrum master was appointed. 

The product owner functions as a representative of the clients' interests, and most importantly manages the product backlog. The product owner is therefore vital for keeping development focused, according the needs of the clients.

The Scrum master's task is to facilitate the work of the development team. He solves problems the team cannot solve themselves, and handles communication with the rest of the organisation. He also coaches the developers in improved organisation and work flow. During this iteration of the \giraf project, the Scrum master functioned most of all as a coordinator, as described in \cref{sec:collab:multiproject:roles}. It may have given an overall better work flow, if the Scrum master had focused on improving the processes, and enforcing these processes, as to prevent the groups from taking more haphazard approaches to development.

\subsection{Intergroup Communication}
Considering how used the involved groups are to working with single-team projects, the communication between the different groups worked very well. 

While a number of formal communication channels were set up early in the project, most importantly the project management tool Redmine, the groups quickly learned to simply visit each other when the need to discuss a common issue arose. This helped preventing a ``them and us'' mentality between groups, and instead helped create a notion of shared responsibility and community. 
