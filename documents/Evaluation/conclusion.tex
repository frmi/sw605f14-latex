This section will answer the question set up in \vagner{Possible ref to Problem formulering}:
\begin{quote}
problemformuleringsspørgsmål, la bla bla bla bla
\end{quote}

The system, when turned over to the clients after the fourth sprint, fulfilled the requirements of being a running and working system.
Furthermore, \launcher has been improved to be as reliable as was possible to make it given the time allocated.\thilemann{Dont like this}
Another requirement was to have all the \giraf applications available through the Google Play Store.
This removes the need for install parties at the end of each sprint for the clients and furthermore enables them to receive updated versions of the applications through the Play Store. \vagner{Do we write anything about that we had install parties with the clients anywhere}
The latter is especially important as a crashing application can be updated easily, instead of the clients having to wait until the end of the next sprint.\\

However, we can also conclude that there are some problems with the \giraf system.\vagner{If there is only one problem, then this sentence should be different.}
The most noticeable of these is the security problem the \giraf system currently has.
The local database provided by \textit{OasisLib} does not include any security regarding access to the database.
Furthermore, the synchronization between the local and remote database initiates by transferring all data from the server to the tablet, before the user can even attempt to log in.
Lastly, the clients required privacy in terms of creating pictograms.
If creating a pictogram as user A, showing that pictogram for user A should be an option.
This has not been implemented and any pictogram created by any user is currently available for all users.\vagner{Is that the only real problem?}\\

Overall, however, the majority of the requirements set forth by the clients have been fulfilled.\vagner{More "ending" tot he conclusion or?}