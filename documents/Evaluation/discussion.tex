This section discusses the development of \launcher, focusing upon elements that could have been done better.\\

First of all, while a class called \lstinline|Constants| does exist, there are still places in the code where ``magic numbers'' are used.
This fx goes for padding in the \lstinline|AppImageView| class when populating a layout with a list of applications.
A good example is seen in \lstinline|HomeActivity| where instantiating a new \lstinline|TranslateAnimation| to animate a view from on position to another.
The constructor \lstinline|TranslateAnimation(0, to, 0, 0)| does not give any indication to what the 0 mean for each parameter.
Instead, constants defined as a members of the class should have been used: \lstinline|new TranslateAnimation(FROM_X, to, FROM_Y, TO_Y)|, to make the code more readable.

Generally seen, this implies that a good code style should have been established before beginning any developments.
\\

Currently, three utility classes exist, each containing between 100 and 300 lines of code.
The common practice when programming in OO-languages, is to contain methods within classes that use them, and generally not create utility classes.
An example of how this could have been done is to have an \lstinline|AppLayout| class, containing all methods related to populating the layout with applications, which should then be used in \lstinline|HomeActivity| and \lstinline|SettingsActivity| classes.\\


Post development, it is concluded that better decisions according to knowing the general best practices for Android development should have been researched further.
The reason being that the implementation of \lstinline|AppViewCreationUtility| is considered inefficient in the way views are handled.
An optimal implementation would extend the \lstinline|GridView| class.
This would simplify the implementation drastically, since it handles spacing between the views which currently is half of the current implementation.

Reflecting on this, the existing code for handling loading applications into view should have been replaced by another implementation, utilizing an adapter as done with \lstinline|SettingsListAdapter| (\cref{sec:settingslistadapter}).\\

Furthermore, testing should have been done more thoroughly regarding developments from both the last group working with \launcher, but also our new additions.
While functional testing were done during the first sprint (\cref{sec:sprint1:testing}), and informal integration testing before each sprint end (\cref{collab:sprintend:three}), neither system- nor unit tests were explored.
As described in \cref{sec:collab:remotedb}, problems were found according to integrate communication with the Remote Database, which indicates that the informal integration testing should have been more extensively.
Lastly, a code review session with another group would have a positive effect on the code.