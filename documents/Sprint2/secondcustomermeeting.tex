\section{Second Customer Meeting}\label{sec:sprint2:secondmeeting}
\thilemann[inline]{Add a reference to the customer meeting in appendix - something like done in the preceeding section...}
The meeting was held the 3rd of April 2014 with two representatives Mette Als Andreasen and Kristine Niss Henriksen from the nursery for children with ASD, called Birken.
The same prototypes was presented as those in the previous meeting (\cref{sec:sprint2:firstmeeting}).
The main purpose of the meeting was to get feedback from a customer that we had not received feedback from before, while also inquiring about some conflicting requirements of this particular customer.

\subsection{Feedback}\vagner{This feedback is based on the presented prototypes, and so, the prototypes described as presented in the section about prototypes, should match these}
The customer was excited about the presented prototypes.

\subsubsection*{Profile Switcher}
The profile selection tool was preferred as a combination of the two options -- as a selector inside each launched application and as a tool-tip from clicking the profile picture in \launcher.

\subsubsection*{Settings}
The settings tool was preferred also as a combination.
Accessing the settings related to each application from inside that particular app was the most important of the two.
Furthermore, pressing the gear\vagner{settingsicon?} while in \launcher should open a full-screen activity, where settings for both \launcher itself and all installed apps could be manipulated.
An important aspect of the settings was the ability to copy settings from one user to another.

\subsubsection*{External (Android) applications}
Lastly, the idea of being able to add apps through \launcher was well received, including apps from Google Play.
The customer especially liked the idea of enabling apps on a `per user' - basis.
This functionality was advised to be in the settings activity found in \launcher.

\subsubsection*{Android buttons}
Apart form the prototype, the solution of overriding the `'Home'', `'Back'' and `'Multitasking'' buttons was accepted, but the customer preferred it to be disabled completely.
It was suggested by the customer to utilized the Android equivalent of iOS' `Restricted Access'', to achieve the preferred result.

\subsection{Clarification of Issues}
One mayor issue regarding conflicting requirements was clarified at this meeting.
Previously, there as a requirement of a black-and-white colorsheme, but also a requirement of changing colors.
It was then clarified that this requirement was related only to the pictograms and not to the program as a whole, meaning the pictogram should in general be black and white and those in color should be able to converted to black and white.
The drawer functionality in \launcher of changing the color of apps was not a requirement and not particularly important for the customer.

\subsection{Communication}
Communication with this customer was more smooth than at the previous meeting, however, there was still comprehension barriers.
Most barriers were related to the customer not understanding what features would be trivial to implement and which would be exceedingly difficult.
Furthermore, questions asked by the group would often have to be reformulated for the customer to understand it completely.
However, the customer often explained how each question was understood, and always pointed out elements they did not understand.
This aided greatly when there were communication problems.

This customer furthermore expressed more concretely what was desired from the application.
Additional questions from the group were still required, but were answered to completion with references to desires.

Overall, the meeting was regarded as a success, with constructive feedback.
