\subsection{Migrating OasisLib}\label{sec:oasismigration}
During sprint one the group developing on the component OasisLib\footnote{OasisLib is a component used by applications in the \giraf project to communicate with the local database.} rewrote their API, which we are using.
They did this because the local database layout was updated to match the remote database layout.
What made the migration difficult was that the old API had been removed completely and there was no documentation on how to migrate to the new version.
Because of the difficulties the migration was postponed to later in the sprint, since we did not want to spend time migrating to the new OasisLib without documentation.

Since other applications in the \giraf project are depended on the \launcher, that is other applications need information about the user, who is passed to the applications when started from \launcher.
We soon started to get requests from the other groups on a new version of the launcher, as they had already migrated to the new OasisLib and local database, see \cref{fig:dependendOasislib}.
This meant that the old \launcher did not support the new local database which they had installed on the devices and therefore crashed.

\inserttexfigure{oasislibdependency}{Illustrates the role of OasisLib in the \giraf project.}{fig:dependendOasislib}

This created some pressure on us to get the process of migrating started.
With still no documentation we started migrating. A lot of the methods in the API had been given new names and there was not consistence in the naming of methods, using both full out spellings and abbreviations, which made it difficult.
For example a method to get the package name for an android application was named \lstinline{getPack()} where another was \lstinline{getProfile()}.

\subsubsection{How to update an API}
Since this project is developed in Java, one has the possibility to use the Java annotation \lstinline{@deprecated}.
This annotation is used in Java when you are updating an API and you want the developers who are using the API to migrate to the new.
By using this annotation the compiler or developing environment warns the developer that a deprecated class, method or field is being used.

According to \citet{deprecatedreference}, a valid reason to deprecate the old API includes ``it is going away in a future release". The deprecation comments helps the developers to chose when to use the new API and could tell which method should be used instead, see \cref{lst:deprecated}.

\begin{lstlisting}[caption={Example of a deprecated method could look like this.}, label={lst:deprecated}]
/**
* Get the ID of the currently logged in guardian.
* @return The guardian ID.
* 
* @deprecated Should no longer be used.
* 	This method has been renamed to create consistence in method naming.
*	One should use getGuardianID instead.
*/
@Deprecated public int getGuardID() {
	return mGuardian.getId();
}
\end{lstlisting}