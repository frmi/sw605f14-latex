\section{First Customer Meeting}\label{sec:sprint2:firstmeeting}
This section focuses on the first customer meeting that was held to determine the viability of the thoughts described in \cref{sec:sprint2:prototypes}.

It was held together with one customer and an external contact on the 1st of April, 2014.
The customer is Drazenko Banjak from the institution called Egebakken, which is a school for children with autism.
The external contact is Pernille Hansen Krogh, the owner of company called Mymo that works with design and decor optimized for learning.


\subsection{Feedback}
Since the contents of the meeting mostly concerned new developments, a slide show was made to support the presentation of the prototypes.
Following that the customers not staying to discuss \giraf at the status meeting, a tablet with the most recent version of \launcher was also brought to get feedback on the changes made during the first sprint (\cref{chap:sprint1}).

\subsubsection{Profile Switcher}

Skifte profil i launcheren og i den enkelte app.

\subsubsection{Settings}
Indstillingerne skal være individuel for hver borger. 
Pernille: mere realistisk at indstille inde i den enkelte app.
Drazenko: vil bare gerne have det virker. 
Drazenko mener det er mest overskueligt at indstille inden i den enkelte app - nervøs for kompleksiteten!

Suggestions based on settings:
En ide kunne være at fjerne muligheden for at bruge andre apps i en hvis periode så fx et spil ikke
er muligt at bruge mens han arbejder med et andet værktøj.
”importer app” knap lige som når du henter billeder ind i et word
dokument


\subsection{Clarification of Issues}

Drawer overflødig hvis ikke farver bruges
Farveskift relevant?

Det er nok en god ide at
snakke med birken omkring hvad de mente med dette farvekrav og med bostedet

Test med brugerne

\subsection{Communication}
At this meeting the communication between the customers and us were quickly characterized as being problematic.
The reason being that Drazenko (customer) did not always quite understand what we were asking for.
The group establishes two possible reasons:
(1) he is lacking technical domain knowledge and does not know the correct terminology\footnote{We quickly established this fact and thus made an effort to use as many non-technical terms as possible.} and (2) he is of foreign extraction and is thus not a profound speaker of danish.
Pernille on the other hand showed great technical understanding even though she, as an external contact, had no previous knowledge of the \giraf project.

The result was that 

how was the communication with customer
- Did they understand what we were asking about?
- Where they able to formulate what they wanted?
