\section{First Client Meeting}\label{sec:sprint2:firstmeeting}
This section focuses on the first client meeting, that is held to determine the viability of the thoughts described in \cref{sec:sprint2:prototypes}.

It is conducted with one client and an external contact on the 1st of April, 2014.
The client is Drazenko Banjak from the institution called Egebakken, which is a school for children with autism.
The external contact is Pernille Hansen Krogh, the owner of the company Mymo, which works with design and décor optimized for learning.
Together, they are referred to as the \textit{clients}.

A full summary of the meeting can be found in \cref{appendix:firstmeeting}.

\subsection{Feedback}\label{sec:firstmeeting:feedback}
Since the content of the meeting is mostly concerned with new developments, a slide show is made to support the presentation of the prototypes.
Following that the clients not were staying to discuss \giraf at the sprint review, a tablet with the most recent version of \launcher is also brought to get feedback on the changes made during the first sprint (\cref{chap:sprint1}).


\subsubsection{Profile Selector}
The profile selector prototypes (\cref{fig:profileselectionlauncherdropdown,fig:profileselectionapppopup}) are presented, and both clients are thrilled about the idea of being able to easily switch profiles from \launcher and each \giraf application individually.

\subsubsection{Settings}
The idea of adding settings management to \launcher, both regarding settings for \launcher itself and for all \giraf applications, is the main idea to be assessed at this meeting.
Therefore the two possibilities, seen in \cref{fig:settingsprototype,fig:appsettingsprototype} are presented.
The clients like the visual look of the prototypes, but have a big concern that the design would add to much complexity to \launcher and that Drazenko's colleagues at Egebakken would be scared of using it.
They like the idea of changing settings from each application for the logged in user, even though we argue that it would impose higher workloads to the guardians if they need to change settings for many users.
A hybrid consisting of both possibilities is suggested to overcome this.

The clients made the following suggestions based on the presentation of the settings management application:

\begin{itemize}
\item Add an external (Android) application to \launcher.
\item Restrict access to use other applications when a guardian has assigned a citizen to work with a specific application.\vagner{Did we forget this?}
\end{itemize}

These suggestions are further considered in \cref{chap:sprint3}.

\subsection{Clarification of Issues}
Although \cref{sec:launcher:drawer} consider improvements to the drawer component in \giraf, a major discussion at the meeting is about whether to keep the drawer or not.
More specifically, Pernille make us rethink if users should be able to colour applications.
If not, it will render the drawer unnecessary.

To overcome this decision, Pernille also suggested us to hold another meeting with Birken, since they are the originator of the requirement.
This is discussed further in \cref{sec:sprint2:secondmeeting}.

\subsection{Communication}
At this meeting the communication between the clients and us are quickly characterized as being problematic.
The reason being that one client does not always quite understand what we are asking about.
The group establishes two possible reasons:
\begin{enumerate}
\item He (Drazenko) is lacking technical domain knowledge and does not know the correct terminology\footnote{We quickly established this fact and thus made an effort to use as many non-technical terms as possible.} or
\item he is of foreign extraction and is thus not a profound speaker of Danish.
\end{enumerate}
Pernille, on the other hand, show great, technical understanding, even though she, as an external contact, has had no previous experience with the \giraf project.
Thus, she does not contribute regarding how \giraf is used 'out in the field', but touches areas that make us rethink previous decisions.

The above facts result in a somewhat vague feedback, in the sense that the client is not always able to clearly formulate his requirements.
Another problem is that when asking a question regarding one topic, feedback is given on another topic which at times are irrelevant in the context of \launcher.

Because of these defects in communication, the meeting is not considered to be a complete success, but useful feedback regarding the settings design has nevertheless been received.