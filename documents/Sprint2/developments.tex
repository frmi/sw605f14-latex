While no developments have been planned for this sprint, a situation arises, forcing us to develop upon \launcher.
The situation is related to the dependencies between the multi-project groups.
Because of these dependencies, this situation will arise in every sprint, however, the developments concerning adjusting to new versions of dependencies will not be described in future sprints.

\subsection{Migrating OasisLib}\label{sec:oasismigration}
At the very end of sprint one, the group responsible for \textit{OasisLib} published a complete rewrite of their API, upon which many groups (including us) are dependent.
They did this because the local database layout has been updated to match that of the remote database.

Migration is made difficult since the old API has been removed completely and no documentation exists on how to migrate to the new API.
Because emphasis in this sprint is given to creating the prototypes and holding the client meetings, added with the fact that the new API comes without documentation, migration is postponed to late in the sprint.

Since other applications in the \giraf suite rely on \launcher passing on information about the currently logged in user, their development effectively comes to a standstill until our implementation is fixed.
Because of this, we soon start to get requests from the other groups on a new version of \launcher, as they have already migrated to the new \textit{OasisLib} and local database, see \cref{fig:dependendOasislib}.
This means that older versions of \launcher do not support the new local database installed on the devices, resulting in \launcher crashing.

\inserttexfigure{oasislibdependency}{Illustrates the role of OasisLib in the \giraf project.}{fig:dependendOasislib}

This adds pressure on us to get the process of migrating to the new version started.
Many of the methods in the API have been given new names and thus are not consistent with the existing API.
Also, their naming scheme is using both full names and abbreviations, which makes it difficult to distinguish their behaviours.
For example, a method to get the package name for an Android application is named \lstinline{getPack()}, where another is named \lstinline{getProfile()}.
The latter being far more descriptive and easier to deduce its functionality from.

\subsubsection{Updating an API}
Since this project is developed in Java, one has the possibility to use the Java annotation \lstinline{@deprecated}.
This annotation is used in Java when you are updating an API and you want the developers who are using the API to migrate to the new one.
By using this annotation, the compiler or developing environment warns the developer that a deprecated class, method or field is being used.

According to \citet{deprecatedreference}, a valid reason to deprecate the old API includes ``it is going away in a future release". 
The deprecation comments helps the developers to chose when to use the new API and could tell which method should be used instead, see \cref{lst:deprecated}.

\begin{lstlisting}[caption={Example of a deprecated method could look like this.}, label={lst:deprecated}]
/**
* Get the ID of the currently logged in guardian.
* @return The guardian ID.
* 
* @deprecated Should no longer be used.
* This method has been renamed to create consistence in method naming.
* One should use getGuardianID instead.
*/
@Deprecated public int getGuardID() {
return mGuardian.getId();
}
\end{lstlisting}

The method of deprecating methods should have been the used method of migrating from the old \textit{OasisLib} API to the new one.