Having completed all remaining task from the first sprint, the second is about making a foundation for future work.
The idea of creating a settings button in \launcher is explored by the means of prototypes and customer meetings.

As the described in \cref{sec:sprint1:review}, the end of the first sprint left us unsure of whether there was more work to be done on \launcher, and if so, what this work could be. 
We decide that it is necessary to arrange a meeting with some of the clients, in part to hear their thoughts on the current version of \launcher, and in part to discover new features.
During the first sprint we informally discussed ideas for new features.
These features include an improved format for the profile selector, and a new Settings application, providing a uniform interface for changing settings across all applications. 
To provide a good foundation for presenting and discussing these ideas with the customers, we design a number of prototype drawings of how our ideas could be realized.

A separate important issue in this sprint is adapting \launcher for new versions its dependencies. 
The \textit{OasisLib} library underwent drastic changes at the end of the first sprint, which unfortunately resulted in several applications, including \launcher, being unable to run.

In summary, the main points of this sprint are:

\begin{itemize}
\item Description of the designed prototypes in \cref{sec:sprint2:analysis}, with summaries of the first meeting in \cref{sec:sprint2:firstmeeting} and the second meeting in \cref{sec:sprint2:secondmeeting}.
\item Description of the migration to the new \textit{OasisLib} in \cref{sec:oasismigration}
\end{itemize}\vagner{Er dette ikke lidt at beskrive hvad vi laver i Design og i developments?}