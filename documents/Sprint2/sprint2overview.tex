As the described in \cref{sec:sprint1:review}, the end of the first sprint left us unsure of whether there was more work to be done on \launcher, and if so, what this work could be. 
We decided that it would be necessary to arrange a meeting with some of the clients, in part to hear their thoughts on the current version of \launcher, and in part to discover new features.
During the first sprint we informally discussed ideas for new features.
These features included an improved format for the profile selector, and a new Settings application, providing a uniform interface for changing settings across all applications. 
To provide a good foundation for presenting and discussing these ideas with the customers, we designed a number of prototype drawings of how our ideas could be realized.\thilemann{Too much use of past tense}

A separate important issue in this sprint was adapting \launcher for new versions its dependencies. The \textit{OasisLib} library underwent drastic changes at the end of the first sprint, which unfortunately resulted in several applications, including \launcher, being unable to run.

In summary, the main points of this sprint are:

\begin{itemize}
\item Description of the designed prototypes in \cref{sec:sprint2:analysis}, with summaries of the first meeting in \cref{sec:sprint2:firstmeeting} and the second meeting in \cref{sec:sprint2:secondmeeting}.
\item Description of the migration to the new OasisLib in \cref{sec:sprint2:developments}
\end{itemize}