\subsection{Conclusion of Customer Meetings}\label{sec:sprint2:conclusionmeetings}

As is apparent from section \ref{sec:sprint2:firstmeeting} and \ref{sec:sprint2:secondmeeting}, the customers were in general enthusiastic about our deviced prototypes.
While smaller adjustments were needed, much of the work done in making the prototypes could be carried over to implementation.

The customer ultimately required:

\begin{itemize}\vagner{doublecheck that this list contains all of the things we got from the customer meetings}
\item A profile selection tool available both from \launcher and from within each individual app
\item A settings tool for all enabled apps accessible from \launcher and a settings tool for each individual app
\item Ability to copy settings from one user profile to another within both versions of the settings tool
\item Ability to enable and disable apps on a per-user basis within the \launcher settings tool
\item Ability to add apps to \launcher from the Google Play store from the \launcher settings tool
\item When attempting the exit an app while \textit{Timer} runs as an overlay, only that app should be accessible untill time runs out
\item While the overridding of the `Home', `Back' and `Multitask' buttons was acceptable, if an `Android version' of the iOS `Restricted Access' functionality can be used to completely block these buttons, that would be preferred
\end{itemize}

Note that other requirements were discovered as well, but were outside of the scope of \launcher.\footnote{As an example can be mentioned the requirement of \textit{Timer} being overlayed upon every app opened from \launcher.}

Furthermore, it was found that the drawer functionality was not a requirement in itself, along with the ability to change colors of the individual apps.

These requirements provided the group with a great backlog for the comming sprint three.