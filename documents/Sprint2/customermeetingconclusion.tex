As apparent from \cref{sec:sprint2:firstmeeting,sec:sprint2:secondmeeting}, the clients are in general enthusiastic about the presented prototypes.
While smaller adjustments are needed to fit their requirements, many of the suggested features are ready to be designed and implemented.

The requirements specified by the clients can be outlined as follows:

\begin{itemize}
\item A profile selection feature available both from \launcher and from within each individual application.
\item A settings activity for all enabled applications accessible from \launcher and inside each individual application.
\item Ability to access Google Play from \launcher settings activity.
\item Ability to add applications from outside the \giraf suite to \launcher.
\item Ability to copy settings from one user profile to another within both versions of the settings activity.
\item Ability to enable and disable applications on a `per user'-basis within \launcher settings activity.
\item When attempting to exit an application while \textit{Tidstager} runs as an overlay, only that application should be accessible until time runs out.
\end{itemize}

Note that requirements outside the scope of \launcher have been discovered as well.
As an example can be mentioned the requirement of \textit{Tidstager} as an overlay for each open application.
Therefore these are not considered any further.

The meetings also uncover unnecessary functionality in form of the drawer component no longer being a requirement, along with the ability to change colours of an specific application on a `per user'-basis.
This discovery also renders our prototype, presented in \cref{fig:profileinfo}, showing profile information in the drawer, impossible to implement and will thus not be considered any further.

The issue of suppressing the Android buttons seems to be essentially unsolvable. 
In spite of several hours of research, we have been unable to find a means for suppressing the \textit{Home} and \textit{Multitask} buttons. 
This intuitively makes sense from a security standpoint, since suppressing these buttons would allow malicious applications to essentially ``hijack'' the device.

These requirements provide the group with a solid backlog for the next sprint.