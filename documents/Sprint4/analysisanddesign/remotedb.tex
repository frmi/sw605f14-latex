\subsection{Remote Database}\label{sec:developments:remotedb}
In order to make the \giraf system even more useful for the clients it was a goal for the semester to implement synchronising the local database with the remote database.
Synchronising with the remote database concerns us since the local database has been integrated into \launcher in this sprint. For clarification on this decision please refer to \Cref{sec:collab:localdbtolauncher} and \Cref{sec:developments:localdbtolauncher} for development details on this matter.

At the moment of writing the synchronisation is only one way. This means that it is only possible for the system to synchronise data from the remote to the local database not the other way around. This was the priority on multi project level, since it meant that the online administration tool could be used to manage the system, and the group responsible the synchronisation did not have time to implement two way synchronisation.\\

The activation of synchronising with the remote database introduced some issues which is described in \Cref{sec:collab:remotedb}.