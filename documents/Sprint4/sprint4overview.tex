The third sprint left a backlog with a number of issues.
These are mostly related to either improving the existing code base or the user experience with settings and \launcher in general.
Therefore, this fourth sprint is dedicated to continuing developments from the third sprint and making \launcher a pleasant program to use.

Since the fourth sprint is the last one of this years development of the project, it is different from the other sprints.
First of all, the clients have put a significant emphasis on receiving working applications that do not crash which was also emphasised by semester coordinator Ulrik Nyman in the first sprint, as mentioned in \cref{sec:sprint1:objectives}.
While we cannot guarantee that \launcher will not crash during the eights months between the end of this sprint and the beginning of the next, we can improve the code to make \launcher as reliable as possible.\\

After development of the application has been completed, a Usability Test is furthermore carried out to gain insight as to what improvements should be done to \launcher next year.