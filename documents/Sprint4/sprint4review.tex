\subsection{Evaluation}

As this is the last sprint of the project, \launcher should be completely done and all functionalities should either have been thoroughly tested or disabled.
There is a large emphasis from the customers on receiving  fully working applications that do not crash.
We fulfil this requirement by the time the deadline for \launcher is reached.

However, one day before the official Sprint End, the \textit{RemoteDatabase} and \textit{OasisLib} groups declared that synchronization between the two was ready to be implemented.
The details are explained in \cref{collab:sprintend:four}.\\

This is an example of where \launcher is ready and our responsibility towards \launcher is fulfilled.
However, \launcher is dependant on the two database groups and as such, on the multi project level, our responsibility towards having all components of \launcher working is not fulfilled, since synchronizing with the remote database was not implemented when we reached our deadline.

This is an important element of the multi project and is further addressed in \cref{}\jesper{Fra vagner: Jesper please read this and please add a reference to the multiproject secion you are writting.}

\subsection{Backlog}

The criteria for a successful sprint end was to have fully working applications, that do not crash.
This meant that some functionality is not added in this sprint, as it has not been tested and debugged and we can therefore not be certain that the application will not crash.

As a result, there are some remaining features to be implemented.
Furthermore, there are several ways the existing code can be improved as well.
A through description of what can be done by the next group working with \launcher can be found in \cref{sec:eval:futureworks}.