As the settings component starts coming together, we decided it is time to evaluate the user interface design, we originally presented to the clients in the second sprint, described in \cref{sec:sprint2:firstmeeting,sec:sprint2:secondmeeting,sec:sprint2:backlog}.

We abstained from testing usability of the original \launcher design earlier, as this had already been done by \citet{launcher2012} in 2012.

At this point, \launcher is still a relatively simple application, and since time is always an issue, we decided on an informal test and evaluation, along the lines of the \textit{Instant Data Analysis} (IDA) method. 

\subsection{The IDA Method}
IDA emphasizes one-day usability test and evaluation sessions, that are meant to be more cost-efficient than the traditional \textit{Video Data Analysis} (VDA) method. 
In its defining article, \citet{idaArticle} describe how the method should reveal most of the problems also revealed through VDA, but in much less time. 

IDA prescribes three players: 
\begin{itemize}
	\item The \textit{test monitor} instructs and observes the test subject during the usability test.
	\item The \textit{data logger} observes the tests, and records events of interest.
	\item Following the actual test session, the \textit{facilitator} runs a brainstorming session with the test monitor and the data logger, where the observed problems are discussed, listed, and rated. 
\end{itemize}

IDA test sessions are based on the think-aloud technique, where the test monitor gives the test subject a number tasks to be carried out with the system. 
The test subject then tries to solve the tasks, while explaining his or her thoughts along the way. 
This makes it easier for the players to identify problematic areas. 
The data logger observes the session, and takes notes based on his observations. 
The facilitator may or may not be present during the test session.

After the test session, the test monitor and the data logger meet with the facilitator for a brainstorming session. 
The observations are presented to the facilitator, who starts compiling a list of usability problems on a white-board. 
Aside from controlling the discussion, it is the job of the facilitator to create an organised list of distinct usability problems, by categorising related observations, and eliminating redundancy. \citet{idaArticle} sets the length of this session to one hour. 

Finally the facilitator compiles a systematic list of problems, with a short description of each.

\subsection{Test Sessions}
We conducted test sessions with four different test subjects:
\begin{itemize}
	\item The first subject has been a client to the \giraf project since it was initiated in 2011, and has tried the various programs several times. 
	She describes herself as ``an okay computer user'', with some experience from the iOS environment, but comparatively little experience with the Android environment. 
	\item The second subject is also a client to the \giraf project, but has little experience in using it, as she feels it has not yet been ready for use. 
	She also describes herself as ``an okay computer user'', with a lot of experience using iOS devices, but little experience with Android. 
	\item The third subject is a school teacher not associated with the \giraf project, but related to one of the developers. 
	She only has superficial knowledge of \giraf, but is very used to Android.
	\item The fourth subject, an optician, is also a relation of one of the developers. 
	As with the third subject, she only knows the \giraf project superficially, but uses Android devices daily.
\end{itemize}

\subsection{Results}
The observations from the four tests were compiled into a list of concrete usability problems. 
We assigned each problem a seriousness rating on the following scale:
\begin{itemize}
	\item A problem is \textit{critical} if the user is unable to finish their task as a result of it.
	\item A problem is \textit{serious} if the user is delayed a significant amount of time, before figuring out how to solve the task.
	\item A problem is \textit{cosmetic} if the user is only delayed for a short amount of time, before figuring out how to solve the task. 
	Cosmetic problems also include issues that did not give the user any problems, but were nonetheless identified as an area of possible improvement.
\end{itemize}

For most problems, we also present a suggested solution, which may be implemented by any future development team.

\subsubsection{Doubt about how the application selection screen works}
\begin{description}
\item{\textbf{Problem}:} Some test subjects had difficulty figuring out how to add applications to a profile's home screen. 
Some tried dragging the icon. 
Some figured out how to mark the application as added, by clicking the icon, but were unsure of whether this solved the task.
\item{\textbf{Rating}:} Critical.
\item{\textbf{Suggested solution}:} It might be more intuitive to have two containers: one with applications not added to the profile, and one with applications added to the profile. 
Applications are then dragged between the two. 
This allows the user to go with their initial impulse to drag the icons instead of clicking them. 
If the containers are named with labels, it should also be easy to figure out the state of an application.
\end{description}


\subsubsection{Difficulty finding Google Play}
\begin{description}
\item{\textbf{Problem}:} Some test subjects had difficulty finding the shortcut to Google Play, when asked to install a certain application. 
They often started with pressing the ``Android Settings'' tab.
Some also correctly chose the ``Apps'' tab, but then went to the section labelled ``Android''. 
\item{\textbf{Rating}:} Serious.
\item{\textbf{Suggested solution}:} We propose a renaming of various tabs. 
The ``Android Settings'' tab should be named ``Device Settings'', to lessen association with Android as a software platform.
The ``GIRAF'' and ``Android'' tabs in the ``Apps'' screen, should be renamed ``GIRAF Apps'' and ``Android Apps'', to make it clearer that these tabs contain application lists, and no settings as such. 
Furthermore, the ``Butik'' tab should be resized to match the other two, to make it more visible.
\end{description}

\subsubsection{Locating the Back button}

\begin{description}
\item{\textbf{Problem}:} The test subjects with little Android experience, initially had difficulty figuring out how to return to a previous screen.
They had to be pointed to the hardware \textit{Back} button.
After this initial confusion, the clients had no problem finding the button again.
\item{\textbf{Rating}:} Serious.
\item{\textbf{Suggested solution}:} Throughout the project period, is has been discussed a number of times to add a software \textit{Back} button to all \giraf applications.
This should completely eliminate any confusion.
Based on our observations, however, we believe that it is only a question of getting used to the existence of a hardware button.
When users have learned to use the hardware button, a software button will only clutter up the user interfaces.
Furthermore, most other Android applications use the hardware button.
\end{description}

\subsubsection{Unsure whether changes are saved}

\begin{description}
\item{\textbf{Problem}:} The first times a test subject changed a setting, they were unsure how to continue, because they could not figure out whether the change had been saved, or if they had to do so themselves.
\item{\textbf{Rating}:} Serious.
\item{\textbf{Suggested solutions}:} We discussed various possibilities for indicating that the changes had been saved, including an on-screen message, a check mark next to the setting, and even a dummy save button.
However, it is an Android, and iOS, norm to save changes as they are made.
We therefore decided that, as with the \textit{Back} button issue, it is a question of the users getting used to the current system.
\end{description}

\subsubsection{Misunderstanding the authentication screen}

\begin{description}
\item{\textbf{Problem}:} Some test subjects did not fully understand how the \authenticationactivity worked, as they tried to keep the camera focused on the QR code, after it had verified the code. 
This made it difficult for them to press the \textit{Login} button, as they awkwardly tried to keep the tablet in position while doing so. 
\item{\textbf{Rating}:} Cosmetic.
\item{\textbf{Suggested solution}:} The camera feed could be turned off as soon as a valid code has been scanned, maybe even turning into the \textit{Login} button, instead of the button appearing under the feed. 
\end{description}