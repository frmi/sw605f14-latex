\subsection{Additional Improvements}

Apart from the improvement described in the above sections, many other improvements are made, some resolving bugs in the code and others improving the visual presentation of \launcher.
However, most of these do not warrant their own sections.
Therefore, a non-exhaustive list of improvements can be found below:

\begin{itemize}
\item When loading applications into a view, the last row often contains fewer applications than the container can accommodate.
This is alleviated by adding invisible, empty view next to the real ones, so the last row appears as the previous ones.
A calculation error is corrected, that accidentally filled the last row with double the amount it could contain, causing all applications to be displayed as significantly smaller views.
\item Two issues arose when closing the application before a \lstinline|ASyncTask| was done with \lstinline|doInBackground|.
The progress bar, which is initiated in \lstinline|onPreExecute|, is never destroyed in \lstinline|onPostExecute| and persists until the activity is restarted.
Furthermore, a textview is referenced in \lstinline|onPostExecute|, which no longer exists at that point, causing an exception.
Complete removal of all progress bars on start up and adding a null pointer check resolved the two problems.
\item The colour of the marking frame used in the ''Apps'' pane, along with several other instances of text throughout the application, is changed to fit the orange coloured theme,
\item A small animation is displayed when launching applications from \lstinline|HomeActivity| now.
\item The padding surrounding the loaded applications was adjusted to be easier on the eye.
\end{itemize}

This concludes the developments made in the fourth sprint.
The following section will present the Usability test done after development.\vagner{Is this too much outro/intro?}