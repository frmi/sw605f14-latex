The entire \giraf multi-project consists of 12 Android applications, two Android libraries, two iOS applications, two web-based applications and a remote database server.
During the period described in this report, around 60 developers divided into 16 teams worked on these components. 
As nearly all of these components either depend on others or are depended upon by others, it was naturally necessary to make the effort somewhat coordinated. 

Allowing the groups to develop their individual components, with no communication between them, will obviously result in an incoherent and chaotic system, that probably will not run at all. 
It is therefore necessary to set up a framework for the cooperation between the groups, and for ensuring that the multi-project as a whole will move towards fulfilling the needs of the clients.

\subsection{Scrum}\label{sec:collab:multiproject:scrum}
By vote, the multi-project development team decided to use Scrum as the overall method of project management. 
We also decided to encourage the use of Scrum in the project groups, but to leave the final decision to the individual groups. 

Larman \cite{larmanAgile} explains that a central part of Scrum is the Scrum Master, who facilitates development by solving problems and enforcing Scrum. 
We decided on a single Scrum Master for the entire project period, to avoid losing gained experience with each changing Scrum Master.

While the Scrum method recommends daily Scrum meetings, we decided to settle with a weekly meeting for the multi-project Scrum. The development tempo is relatively low as we also have to follow courses. 
Some of these courses are electives, so it will also be difficult to gather every group at the same time every day.

A short evaluation of the use of Scrum on the multi-project level can be read in \cref{sec:eval:multiproject:scrum}.

\subsection{Sprints}
We decided to work in sprints. 
Changing requirements is the main argument against the traditional ``single-sprint'' model. 
The resulting system may not live up to the clients' expectations, so it is important to regularly align the visions of the clients and the developers. 

As a unit for measuring time in the sprints, we chose ``half days'', which last 3-4 hours. 
This unit fits well into our schedule, with relation to course modules.
The available half days were divided into four sprints:

\begin{enumerate}
\item First sprint starts 24-02-2014 and ends 19-03-2014.
\item Second sprint starts 20-03-2014 and ends 14-04-2014.
\item Third sprint starts 15-04-2014 and ends 07-05-2014.
\item Fourth sprint starts 08-05-2014 and ends 27-05-2014.
\end{enumerate}

At the end of each sprint all groups should present and demonstrate their topic at a sprint end meeting with the clients.

\subsection{Roles and Responsibilities}
\label{sec:collab:multiproject:roles}
A large number managerial tasks were delegated to members of the different groups. 
The most notable were:
\begin{itemize}
	\item The \textit{Scrum Master} solved tasks mainly related to coordination. 
	His chief task was to chair the weekly Scrum meetings, where the groups summed up their progress, and brought up problems and suggestions relevant to other groups. 
	He also had some tasks related to project-wide initiatives and problems, e.g. readying the applications for release to Google Play.
	\item The \textit{Client Contact} took care of all communication with the clients, giving the them a single contact point, and made sure that clients were not overburdened with individual requests and questions from the 16 groups. 
	\item The \textit{Sprint End Specialist} was appointed after the problematic first sprint review meeting, described in \cref{sec:collab:sprintend}. 
	His task was to plan and organise the sprint review meetings, where the groups' progress is presented to the clients.
\end{itemize}
