As we are using the Scrum method of development for the entire \giraf project, a sprint review meeting was planned for the end of each iteration. 
\citet[p. 71]{larmanAgile} describes this type of meeting as follows:
\begin{quote}
``At the end of each iteration, there is a review meeting (maximum of four hours)
hosted by the Scrum Master.
The team, Product Owner, and other stakeholders attend.
There is a demo of the product.

Feedback and brainstorming on future directions is encouraged, but no commitments are made during the meeting.
Later at the next Sprint Planning meeting, stakeholders and the team make commitments."
\end{quote}

\subsection{The First Sprint Review Meeting}
At the meeting following the first sprint, the Scrum master of the \giraf project started with a short presentation, and then gave the floor to the groups.
They spent five minutes each demonstrating their progress to the stakeholders. 
After the demonstrations, the stakeholders left, and the groups that would not continue working on the same project (\giraf application) for the next sprint, proceeded to distribute these projects among them, in preparation for the second sprint.

We do not consider this sprint review meeting a success, due to the following reasons:
\begin{itemize}
	\item It was a one-way meeting, where the stakeholders were presented with the progress, but were not encouraged to give any feedback. 
	As a result, the development teams had no gain from the meeting, and were ill prepared for the second sprint. 
	This problem was exacerbated by the lack of the sprint planning meetings required in Scrum to start a new sprint.
	As \citet[p. 70]{larmanAgile} describes, during the sprint planning meetings, goals are re-prioritized and a new sprint backlog is created. 
	\item Several of the demonstrations had technical problems, such as applications crashing in the emulator used for the demonstration.
	This surprised the presenting groups, and thus they were unable to demonstrate their actual progress.
	A related problem was that several groups did not direct their presentation towards the costumers, and therefore focused on technical details, using a very technical vocabulary.
	The demonstration should be better prepared, with more focus on the actual goal of the meeting.
	\item Not all stakeholders were able to attend, as the meeting was planned at a relative late date beforehand.
	As the main goal of the meeting is for the stakeholders to review the progress of the project, this is not optimal.
\end{itemize}

These problems were discussed after the conclusion of the meeting. 
It was suggested that a possible root problem was the \giraf Scrum master being overworked with planning this meeting along with his other responsibilities.
It was decided to, at a later date, discuss the possibility of designating another person as responsible for planning and conducting the sprint review meeting, and possibly the sprint planning meetings as well. Eventually this role was given to a member of this project group.

\subsection{Sprint Review Planning}
\label{collab:sprintend:planning}
To counter the above mentioned points, the new sprint end specialist wrote a directive on how Sprint Review meetings should be performed. Main points in this document include:
\begin{itemize}
	\item The clients should be encouraged to give feedback continually during the meeting, hopefully giving the groups an idea of where their projects are still lacking. Furthermore, the second half of the meeting will consist of a planning phase, where the clients have the opportunity to re-prioritise the goals of the upcoming tasks, or add goals if something important has come up during the meeting. Scrum suggests building the new 	sprint backlog from scratch in cooperation with the clients. We however, see no reason to keep up to nine clients and 60 students in a room for several hours to discuss the sprint backlog, which has to be large enough to keep every student working throughout the upcoming sprint. Instead, one student will compile a preliminary suggestion from the backlog, based on what each group think is the next step for their project. We then give the clients the opportunity to influence this backlog.
	\item The groups are given deadlines for releasing their projects in the versions to be used for the demonstration. Projects that other projects depend upon are to be released to the other groups at least three working days before the sprint review. This gives the other groups a chance to adapt their own projects to the new dependencies. Groups who wish to demonstrate their product to the clients, are to release their final versions at least one working day in advance. This gives the sprint end specialist a chance to collect the applications on a tablet, and let the groups test whether their application works as expected on this tablet. 
	\item To keep the demonstrations on a level that the clients understand, the groups are asked to base their demonstrations on user stories along the lines of ``You are now able to ... by doing ...''. This should prevent the groups from becoming technical in their presentations, and from discussing topics irrelevant to the clients, such as how many hours were spent debugging. 
	\item The dates of the sprint review are advertised to clients well in advance, with a program for the upcoming meeting being sent out at least a week in advance, which should ensure that as many clients as possible are able to attend. 
\end{itemize}

\subsubsection{Install Parties}
At each sprint review meeting, the clients have to bring their Android devices to update the installed \giraf applications to their latest version with the new developments accomplished during the particular sprint.
This is not seen as being an efficient method of distributing updates.
As mentioned in \Cref{sec:collab:localdbtolauncher}, the apps should be avialable through Google Play since this is the correct method to distribute applications to Android devices.
If the applications are distributed through Google Play, the users will automatically get the updates installed on their devices.

\subsection{The Second Sprint Review Meeting}
\label{collab:sprintend:two}
It is difficult to evaluate on the second meeting, as the clients left during a break, due to a misunderstanding. Furthermore, the sprint end specialist was was unable to participate in neither the preparations, nor the meeting itself, so no detailed evaluation of the meeting was conducted. The meeting itself ran into problems, when the clients left the university during an early break, due to a misunderstanding. 

\subsection{The Third Sprint Review Meeting}
\label{collab:sprintend:three}
The third meeting was the first to be executed fully in accordance with the new meeting directive. In the final days before the meeting, the sprint end specialist spent significant amounts of time checking that the various groups finished their demo versions and that the groups tested their applications along with the other applications, in what can be described as an informal integration and system test. 

During the meeting, the sprint end specialist tried to involve the clients in discussions, by asking them open-ended questions. This achieved the affect of making the clients more open to present their thoughts on the various applications. The presentations were much more minded towards the clients, and at a later evaluation of the meeting, it was agreed that the only problem was the presentation skills of some groups, who had some issues with mumbling etc.

The planning phase could be characterised by limited success, as the clients for the most part accepted the suggested backlogs with no comments. A single project had more task suggestions than the group would be able to finish, and the clients were asked directly to opt one out, which they hesitantly did. The problem may have been that they felt awkward telling the groups what to do, and what not to do.

\subsection{The Fourth Sprint Review Meeting}
\label{collab:sprintend:four}
The fourth meeting was somewhat different, as it marked the end of the last sprint, and thus no further developments would be made afterwards. Therefore the time was spent with presenting all the developed applications, while the sprint planning phase was left out. 

The preparations for this meeting became somewhat more chaotic, as it was decided to try making synchronisation between the remote and local databases work. While all deadlines had passed, the day before the meeting, the remote database was declared ready for this step. We decided to go through with it and implement database synchronisation. The expected technical problems showed up, and the overall system became somewhat more unstable, but due to the addition of actual pictograms, we believe it is more useful to the clients than the stable, but empty version we had ready before the integration of database synchronisation. The issues and decisions regarding this, is described in detail in \cref{sec:collab:remotedb}.

