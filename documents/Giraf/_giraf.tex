\chapter{GIRAF}\label{chap:giraf}

\section{Introduction}\label{sec:giraf:introduction}
\jesper[inline]{Fix this section in relation to the introduction.}
\giraf (Graphical Interface Resources for Autistic Folk) is a suite of applications designed for aiding the mental development of people with autism, mainly children. The applications are developed iteratively at Aalborg University's Department of Computer Science, by software engineering students on the final semester of their bachelor.

The project started in 2011, and has subsequently evolved with each class of students bringing new ideas to the project, and improving on the existing ones.
\giraf is developed in cooperation with Aalborg Municipality, which has several institutions specialising in children and adults with autism. 
These institutions rely heavily on paper-based tools for their work. 
The idea behind \giraf is to organize and streamline these tools, by implementing them on tablet computers.

\section{\giraf Components}\label{sec:giraf:components}
\jesper[inline]{Write a very short summary on each GIRAF component, so we can refer to it when mentioning these later on.}

\section{Multi-Project Development Method}\label{sec:giraf:development}
The \giraf development team of 2014 consisted of about 60 students, organised into 16 project groups. As previously described, some of the \giraf components are heavily dependent on others. 

\thilemann{Something about the development method, and especially how sprints are used. This is needed before describing the sprints in subsequent chapters.}

	\thilemann[inline]{Below is taken from statusmeeting summary on Redmine wiki. Remember to add sprint 4 end.}

Sprint
Sprint unit is measured in half days.

There is 94 half days left from 24-02-2014 to 27-05-2014 measured with Semantics and Verification as optional course.

\begin{enumerate}
\item First sprint starts 24-02-2014 and ends 19-03-2014.\\
Sprint duration: 18 half days.
\item Second sprint starts 20-03-2014 and ends 14-04-2014.\\
Sprint duration: 25 half days.
\item Third sprint starts 15-04-2014 and ends 07-05-2014.\\
Sprint duration: 24 half days.
\item Fourth sprint starts 08-05-2014 and ends 27-05-2014.\\
Sprint duration: 27 half days.
\end{enumerate}

At the end of each sprint all groups should present and demonstrate their topic.
It is strongly recommended that everybody participates in these events. Thus you as well as the invited customers and supervisors gets and overview of the project status.