\chapter{GIRAF}\label{chap:giraf}
\giraf  stands for \textit{Graphical Interface Resources for Autistic Folk} and is used to describe the entire multiproject as an entity.
The suite of \giraf applications are developed iteratively at Aalborg University's Department of Computer Science, by software engineering students on the final semester of their bachelor.
The project was initiated in 2011 by Ulrik Nyman, Associate Professor at Aalborg University.
The project has subsequently evolved with each class of students bringing new ideas to the project, and improving on the existing ones.
Furthermore, the requirements from the customers has changed over time.

Overall, the multi-project is designed for various kinds of persons.
Most importantly being the ``citizens" utilizing the capabilities of each application, where citizen refers to either a child or an adult person diagnosed with ASD.
Each citizen is supervised by another adult, referred to as being a ``guardian".
This person is typically employed in one of the institutions affiliated with the \giraf project.
Thus, when referring to both, the term ``user" is used.

This chapter gives a short introduction to the \giraf multi-project and the software engineering concepts used during its development.

\section{\giraf Applications}\label{sec:giraf:applications}
The \giraf suite consists of several components, not the least a multitude of front end applications, providing various features to the users. All front end applications are described in \Cref{sec:giraf:applications:frontend}. 
The suite however, also includes a number of back end components, allowing sharing and controlling data across the range of applications. 
The back end components are described below.

\subsection{Back End Components}
\label{sec:giraf:applications:backend}
Central to most of the Android applications is \textit{OasisLib}, which provides both models for the entire problem domain, and interfaces for retriving data from the local database. Every device has its own local database, that regularly synchronises with a single remote database\thilemann{Does it? Only at the end of sprint four this was accomplished.}. Finally, \textit{\giraf Components} contains common user interface components with a uniform look-and-feel. 

\section{GIRAF Architecture}
\label{sec:giraf:architecture}
% Sådan er multiprojektet bygget op.
The \giraf system is illustrated in \Cref{fig:girafLayers} and as illustrated in consists of several layers. From the top we have applications such as those described in \Cref{sec:giraf:applications:frontend} which communicate with a local database on the device through the library \textit{OasisLib} which is a part of the \giraf backend.

The local database is responsible for communicating with the remote database to synchronize local or remote changes in the system.

To ensure that the layout in all applications has the same look and feel the library \textit{GIRAF Components} has been introduced.
This library consists of multiple graphical components and provide common services to applications.
A common service which applications needs is when a guardian has to choose which citizen settings or an application should be loaded for. Then \textit{GIRAF Components} provides a component to show a list of citizens which the guardian can choose from. By providing this component, the process of choosing a citizen will be the same in all applications in the \giraf system.

\insertfigure{width=0.8\textwidth}{girafLayers}{Illustration of the layers and dependencies in the \giraf application suite.}{fig:girafLayers}

\section{Multi-Project Development Method}\label{sec:giraf:development}
The \giraf development team of 2014 consists of about 60 students, organised into 16 project groups. 
As previously described, some of the \giraf components are heavily dependent on others. 
Allowing the groups develop their individual components, with no communication between them, will obviously result in an incoherent and chaotic system, that probably will not run at all. 
It is therefore necessary to set up a framework for the cooperation between the groups, and for ensuring that the multi-project as a whole will move towards fulfilling the needs of the clients.

\subsection{Scrum}
By vote, the multi-project development team decided to use Scrum at the overall method of project management. 
We also decided to encourage the use of Scrum in the project groups, but to leave the final decision to the individual groups. 

Larman \cite{larmanAgile} explains that a central part of Scrum is the Scrum Master, who facilitates development by solving problems and enforcing Scrum. We decided on a single Scrum master for the entire project period, to avoid losing gained experience with each changing Scrum master.

While the Scrum method recommends daily Scrum meetings, we decided to settle with a weekly meeting for the multi-project Scrum. The development tempo is relatively low as we also have to follow courses. Some of these courses are electives, so it will also be difficult to gather every group at the same time every day.\jesper{Remember to add reference to multi project section}

\subsection{Sprints}
We decided to work in sprints. Changing requirements is the main argument against the traditional ``single-sprint'' model. The resulting system may not live up to the client's expectations, so it is important regularly align the visions of the clients and the developers. This concept is systematised with multiple sprints. The individual components may also experience changes in what they require of each other.

As a unit for measuring time in the sprints, we chose ``half days'', which last 3-4 hours. This unit fits well into our schedule, with relation to course modules.

The available half days were divided into four sprints:

\begin{enumerate}
\item First sprint starts 24-02-2014 and ends 19-03-2014.
\item Second sprint starts 20-03-2014 and ends 14-04-2014.
\item Third sprint starts 15-04-2014 and ends 07-05-2014.
\item Fourth sprint starts 08-05-2014 and ends 27-05-2014.
\end{enumerate}

At the end of each sprint all groups should present and demonstrate their topic at a sprint end meeting with the clients.
