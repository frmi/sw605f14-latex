\chapter{GIRAF}\label{chap:giraf}

\section{Introduction}\label{sec:giraf:introduction}
\jesper[inline]{Fix this section in relation to the introduction.}
\giraf (Graphical Interface Resources for Autistic Folk) is a suite of applications designed for aiding the mental development of people with autism, mainly children. The applications are developed iteratively at Aalborg University's Department of Computer Science, by software engineering students on the final semester of their bachelor.

The project started in 2011, and has subsequently evolved with each class of students bringing new ideas to the project, and improving on the existing ones.
\giraf is developed in cooperation with Aalborg Municipality, which has several institutions specialising in children and adults with autism. 
These institutions rely heavily on paper-based tools for their work. 
The idea behind \giraf is to organize and streamline these tools, by implementing them on tablet computers.

\section{\giraf Components}\label{sec:giraf:components}
\jesper[inline]{Write a very short summary on each GIRAF component, so we can refer to it when mentioning these later on.}

\section{Multi-Project Development Method}\label{sec:giraf:development}
The \giraf development team of 2014 consists of about 60 students, organised into 16 project groups. As previously described, some of the \giraf components are heavily dependent on others. Allowing the groups develop their individual components, with no communication between them, will obviously result in an incoherent and chaotic system, that probably will not run at all. It is therefore necessary to set up a framework for the cooperation between the groups, and for ensuring that the multi-project as a whole will move towards fulfilling the needs of the clients.

\subsection{Scrum}
By vote, the multi-project development team decided to use Scrum at the overall method of project management. 
We also decided to encourage the use of Scrum in the project groups, but to leave the final decision to the individual groups. 

Larman \cite{larmanAgile} explains that a central part of Scrum is the Scrum Master, who facilitates development by solving problems and enforcing Scrum. We decided on a single Scrum master for the entire project period, to avoid losing gained experience with each changing Scrum master.

While the Scrum method recommends daily Scrum meetings, we decided to settle with a weekly meeting for the multi-project Scrum. The development tempo is relatively low as we also have to follow courses. Some of these courses are electives, so it will also be difficult to gather every group at the same time every day.

\subsection{Sprints}
We decided to work in sprints. Changing requirements is the main argument against the traditional ``single-sprint'' model. The resulting system may not live up to the client's expectations, so it is important regularly align the visions of the clients and the developers. This concept is systematised with multiple sprints. The individual components may also experience changes in what they require of each other.

As a unit for measuring time in the sprints, we chose ``half days'', which last 3-4 hours. This unit fits well into our schedule, with relation to course modules.

The available half days were divided into four sprints:

\begin{enumerate}
\item First sprint starts 24-02-2014 and ends 19-03-2014.
\item Second sprint starts 20-03-2014 and ends 14-04-2014.
\item Third sprint starts 15-04-2014 and ends 07-05-2014.
\item Fourth sprint starts 08-05-2014 and ends 27-05-2014.
\end{enumerate}

At the end of each sprint all groups should present and demonstrate their topic at a sprint end meeting with the clients.