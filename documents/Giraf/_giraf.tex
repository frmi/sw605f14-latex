\chapter{GIRAF}\label{chap:giraf}
The suite of \giraf applications are developed iteratively at Aalborg University's Department of Computer Science, by software engineering students on the final semester of their bachelor.
The project was initiated in 2011 by Associate Professor at Aalborg University, Ulrik Nyman.
The project has subsequently evolved with each class of students bringing new ideas to the project, and improving on the existing ones.

This chapter gives a short introduction to the \giraf multi-project and the software engineering concepts used during its development.

\section{\giraf Applications}\label{sec:giraf:applications}
The \giraf suite consists of many kinds of applications, ranging over administrative tools on tablet and web, games, back-end such as database access and others more specific to the task of helping the children.
This section briefly describes the most central applications in the \giraf suite.

\paragraph{\launcher}
is the only application directly associated with the name \giraf.
It is the main entry-point for accessing any other application (except web-based tools) on Android devices.
The purpose of \launcher is to shield the users from interacting with the Android system.

\paragraph{OasisLib}
\thilemann{Write something clever about OasisLib...}

\paragraph{\giraf Components}
is not an application in \giraf, but an internal library that are used to render User Interface components defining the look and feel of the project.
It is thus used by all Android applications.

\paragraph{Parrot}
is an application that helps speech impaired users communicate (and learn how to speak the words) by the use of pictograms\footnote{Small icons resembling physical objects, fx an icon of a table.}.
By arranging sequences of pictograms the users are able to form sentences that can be heard of a guardian, communicating the message of the user.

\paragraph{Remote Database}
\thilemann{Add how connected with LocalDB}

\paragraph{Local Database}
\thilemann{Add how connected with RemoteDB}

\paragraph{Cars (game)}
is one of the games in \giraf.
It aims to strengthen the speech of its users by using their voice to move a car traveling horizontally across the screen vertically by volume level.

%\paragraph{Other \giraf Applications}
%Timer, Train, Tortoise, Parrot, Sequence, Picto Search, CAT, \giraf Admin.

\thilemann{Add a dependency graph of the applicatoins, NICE!}

\section{Multi-Project Development Method}\label{sec:giraf:development}
The \giraf development team of 2014 consisted of about 60 students, organised into 16 project groups. As previously described, some of the \giraf components are heavily dependent on others. 

\thilemann{Something about the development method, and especially how sprints are used. This is needed before describing the sprints in subsequent chapters.}

	\thilemann[inline]{Below is taken from statusmeeting summary on Redmine wiki. Remember to add sprint 4 end.}

Sprint
Sprint unit is measured in half days.

There is 94 half days left from 24-02-2014 to 27-05-2014 measured with Semantics and Verification as optional course.

\begin{enumerate}
\item First sprint starts 24-02-2014 and ends 19-03-2014.\\
Sprint duration: 18 half days.
\item Second sprint starts 20-03-2014 and ends 14-04-2014.\\
Sprint duration: 25 half days.
\item Third sprint starts 15-04-2014 and ends 07-05-2014.\\
Sprint duration: 24 half days.
\item Fourth sprint starts 08-05-2014 and ends 27-05-2014.\\
Sprint duration: 27 half days.
\end{enumerate}

At the end of each sprint all groups should present and demonstrate their topic.
It is strongly recommended that everybody participates in these events. Thus you as well as the invited customers and supervisors gets and overview of the project status.