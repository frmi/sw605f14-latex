\chapter{GIRAF}\label{chap:giraf}

\section{Introduction}\label{sec:giraf:introduction}
\thilemann{Write something about \giraf (consider introduction)}

\section{Multi-Project Development Method}\label{sec:giraf:development}
\thilemann{Something about the development method, and especially how sprints are used. This is needed before describing the sprints in subsequent chapters.}

	\thilemann[inline]{Below is taken from statusmeeting summary on Redmine wiki. Remember to add sprint 4 end.}

Sprint
Sprint unit is measured in half days.

There is 94 half days left from 24-02-2014 to 27-05-2014 measured with Semantics and Verification as optional course.

\begin{enumerate}
\item First sprint starts 24-02-2014 and ends 19-03-2014.\\
Sprint duration: 18 half days.
\item Second sprint starts 20-03-2014 and ends 14-04-2014.\\
Sprint duration: 25 half days.
\item Third sprint starts 15-04-2014 and ends 07-05-2014.\\
Sprint duration: 24 half days.
\item Fourth sprint starts 08-05-2014 and ends 27-05-2014.\\
Sprint duration: 27 half days.
\end{enumerate}

At the end of each sprint all groups should present and demonstrate their topic.
It is strongly recommended that everybody participates in these events. Thus you as well as the invited customers and supervisors gets and overview of the project status.