\chapter{GIRAF}\label{chap:giraf}
\thilemann{Add GIRAF abv.}
The suite of \giraf applications are developed iteratively at Aalborg University's Department of Computer Science, by software engineering students on the final semester of their bachelor.
The project was initiated in 2011 by Associate Professor at Aalborg University, Ulrik Nyman.
The project has subsequently evolved with each class of students bringing new ideas to the project, and improving on the existing ones.

This chapter gives a short introduction to the \giraf multi-project and the software engineering concepts used during its development.

\section{\giraf Applications}\label{sec:giraf:applications}
The \giraf suite consists of many kinds of applications, ranging over administrative tools on tablet and web, games, back-end such as database access and others more specific to the task of helping the users.
This section briefly describes the most central applications in the \giraf suite.
\thilemann{Should we rename the applications listed below, and should we add some other? Fx metadata?}

\paragraph{\launcher}
is the only application directly associated with the name \giraf.
It is the main entry-point for accessing any other application (except web-based tools) on Android devices.
The purpose of \launcher is to shield the users from interacting with the Android system.

\paragraph{OasisLib}
is a general component used to query the (remote) database.
Hence, it queries the database by sending SQL commands and returns the data de-serialized as Java Objects.
OasisLib is central to every \giraf application working with data and is thus a dependency in almost every application.

\paragraph{Database(s)}
is providing each user with settings and other data to be used in the \giraf applications.
Since all applications are intended to work in offline-mode (since not all tablets have a mobile broadband connection), \giraf comprises of a local database holding all user data and a separate remote database that is queried whenever Internet access is available to synchronize the local database. 

\paragraph{\giraf Components}
is not an application in \giraf, but an internal library that is used to render User Interface components defining the look and feel of the project.
It is thus used by all Android applications.

\paragraph{Parrot}
is an application that helps speech impaired users communicate (and learn how to speak the words) by the use of pictograms\footnote{Small icons resembling physical objects, fx an icon of a table.}.
By arranging sequences of pictograms the users are able to form sentences that can be heard of a guardian, communicating the message of the user.\thilemann{Please verify this}

\paragraph{Cars (game)}
is one of the games in \giraf.
It aims to strengthen the speech of its users by using their voice to move a car, traveling horizontally across the screen, vertically by volume level.

%\paragraph{Other \giraf Applications}
%Timer, Train, Tortoise, Parrot, Sequence, Picto Search, CAT, \giraf Admin.

\insertfigure{width=\textwidth}{girafLayers}{Illustration of the layers and dependencies in the \giraf application suite.}{fig:girafLayers}

\section{Multi-Project Development Method}\label{sec:giraf:development}
The \giraf development team of 2014 consists of about 60 students, organised into 16 project groups. 
As previously described, some of the \giraf components are heavily dependent on others. 
Allowing the groups develop their individual components, with no communication between them, will obviously result in an incoherent and chaotic system, that probably will not run at all. 
It is therefore necessary to set up a framework for the cooperation between the groups, and for ensuring that the multi-project as a whole will move towards fulfilling the needs of the clients.

\subsection{Scrum}
By vote, the multi-project development team decided to use Scrum at the overall method of project management. 
We also decided to encourage the use of Scrum in the project groups, but to leave the final decision to the individual groups. 

Larman \cite{larmanAgile} explains that a central part of Scrum is the Scrum Master, who facilitates development by solving problems and enforcing Scrum. We decided on a single Scrum master for the entire project period, to avoid losing gained experience with each changing Scrum master.

While the Scrum method recommends daily Scrum meetings, we decided to settle with a weekly meeting for the multi-project Scrum. The development tempo is relatively low as we also have to follow courses. Some of these courses are electives, so it will also be difficult to gather every group at the same time every day.

\subsection{Sprints}
We decided to work in sprints. Changing requirements is the main argument against the traditional ``single-sprint'' model. The resulting system may not live up to the client's expectations, so it is important regularly align the visions of the clients and the developers. This concept is systematised with multiple sprints. The individual components may also experience changes in what they require of each other.

As a unit for measuring time in the sprints, we chose ``half days'', which last 3-4 hours. This unit fits well into our schedule, with relation to course modules.

The available half days were divided into four sprints:

\begin{enumerate}
\item First sprint starts 24-02-2014 and ends 19-03-2014.
\item Second sprint starts 20-03-2014 and ends 14-04-2014.
\item Third sprint starts 15-04-2014 and ends 07-05-2014.
\item Fourth sprint starts 08-05-2014 and ends 27-05-2014.
\end{enumerate}

At the end of each sprint all groups should present and demonstrate their topic at a sprint end meeting with the clients.
