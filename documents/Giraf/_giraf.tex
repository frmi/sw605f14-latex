\chapter{GIRAF}\label{chap:giraf}
\giraf  stands for \textit{Graphical Interface Resources for Autistic Folk} and is used to describe the entire multiproject as an entity.
The suite of \giraf applications are developed iteratively at Aalborg University's Department of Computer Science, by software engineering students on the final semester of their bachelor.
The project was initiated in 2011 by Ulrik Nyman, Associate Professor at Aalborg University.
The project has subsequently evolved with each class of students bringing new ideas to the project, and improving on the existing ones.
Furthermore, the requirements from the customers has changed over time.

Overall, the multi-project is designed for various kinds of persons.
Most importantly being the ``citizens" utilizing the capabilities of each application, where citizen refers to either a child or an adult person diagnosed with ASD.
Each citizen is supervised by another adult, referred to as being a ``guardian".
This person is typically employed in one of the institutions affiliated with the \giraf project.
Thus, when referring to both, the term ``user" is used.

This chapter gives a short introduction to the \giraf multi-project.
For a description of the development methodology employed, see \cref{sec:collab:multiproject}.

\section{\giraf Applications}\label{sec:giraf:applications}
The \giraf suite consists of several components, not the least a multitude of front end applications, providing various features to the users. 
The suite however, also includes a number of back end components, allowing sharing and controlling data across the range of applications. 
These parts of the system will be described in the following.

\subsection{Front End Applications}
\label{sec:giraf:applications:frontend}
Below is a list of the applications included in the \giraf suite.

\paragraph{\launcher}
provides an interface for accessing the other tablet applications in a controlled environment, that is easy to use for both guardians and citizens. 
Through \launcher, the guardians should be able to control what applications the citizens should be allowed to use. In the application interface, \launcher is referred to as \giraf.

\paragraph{Sekvens}
allows users to build sentences from pictograms, a central activity in the lives of the citizens \giraf is made to service. Many citizens with autism, especially young children, have difficulty formulating sentences in speech. A central feature of \textit{Sekvens} is also the possibility to save often used sequences of pictograms.

\paragraph{Pictooplæser}
is similar to \textit{Sekvens}, but is focused on building more ad hoc sentences, which the application is then able to read aloud using either an existing recording, or an online text-to-speech tool. 

\paragraph{Kategoriværktøjet}
allows guardians to manage the categories into which the pictograms are organised, including adding and deleting categories and subcategories.

\paragraph{Oasis App}
is an administration tool for manipulating the user information in the database. 

\paragraph{Pictosearch}
is not used as a standalone application, but provides other applications with a common interface for searching through the device's collection of pictograms.

\paragraph{Pictotegner}
allows the user to create their own pictograms with basic graphics tools, such as a free-hand pen tool, a rectangle tool, a circle tool etc. 

\paragraph{Livshistorier}
is also similar to \textit{Sekvens}, but is created specifically for saving pictogram sequences that contain instructions for the citizen's everyday life, e.g. instructions on how to use the bathroom.

\paragraph{Ugeplan}
allows the users to build weekly activity schedules using pictograms. Autistic citizens thrive best in highly strutured environments with regularly scheduled days.

\paragraph{Tidstager}
provides the ability to time an activity, allowing a guardian to let a citizen use an application, e.g. a game, for a set amount of time. When the time runs out, the device locks, forcing the citizen to move on to the next scheduled activity.

\paragraph{Stemmespillet}
is a game, where the citizen controls a car by varying the volume of his or her voice. 

\paragraph{Kategorispillet}
is a game, where the citizen unloads pictograms from a train. The pictograms must be unloaded at different stations, where each station accepts a certain category of pictograms.

\paragraph{Web Ugeplan}
is a web-version of \textit{Ugeplan}, specifically design for large touchscreens. A few costumers had access to TV-size touchscreen, which they specifically wished to use for their week schedules.

\paragraph{Webadmin}
is a web-based administration tool for manipulating the user information in the database.

\subsection{Back End Components}
\label{sec:giraf:applications:backend}
Central to most of the Android applications is \textit{OasisLib}, which provides both models for the entire problem domain, and interfaces for retriving data from the local database. Every device has its own local database, that regularly synchronises with a single remote database\thilemann{Does it? Only at the end of sprint four this was accomplished.}. Finally, \textit{\giraf Components} contains common user interface components with a uniform look-and-feel. 

\section{GIRAF Architecture}
\label{sec:giraf:architecture}
% Sådan er multiprojektet bygget op.
The \giraf system is illustrated in \Cref{fig:girafLayers} and as illustrated in consists of several layers. From the top we have applications such as those described in \Cref{sec:giraf:applications:frontend} which communicate with a local database on the device through the library \textit{OasisLib} which is a part of the \giraf backend.

The local database is responsible for communicating with the remote database to synchronize local or remote changes in the system.

To ensure that the layout in all applications has the same look and feel the library \textit{GIRAF Components} has been introduced.
This library consists of multiple graphical components and provide common services to applications.
A common service which applications needs is when a guardian has to choose which citizen settings or an application should be loaded for. Then \textit{GIRAF Components} provides a component to show a list of citizens which the guardian can choose from. By providing this component, the process of choosing a citizen will be the same in all applications in the \giraf system.

\insertfigure{width=0.8\textwidth}{girafLayers}{Illustration of the layers and dependencies in the \giraf application suite.}{fig:girafLayers}

