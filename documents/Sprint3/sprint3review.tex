\subsection{Evaluation}
Implementing the settings was the primary task for the third sprint.
The task consisted of several subtasks, which have been completed:

\begin{itemize}
\item It is possible to scale applications up and down and disable the start up animation.
\item It is possible to add and remove both \giraf and Android applications on a `per user'-basis.
\item It is possible to access Google Play and the native Android settings from \settingsactivity.
\item It is possible to access the settings of other \giraf applications through \settingsactivity.
\end{itemize}

This means the task has been solved and the sprint has been a success. 

We decided not to present our progress at this sprint review meeting. 
While it technically goes against Scrum practice to not demonstrate the current state of the project to the clients at each sprint review, \settingsactivity in its current state is essentially identical to the prototypes previously presented to the clients.
Furthermore the clients do not have the same urge to monitor progress, as paying clients might have.

Sprint review is once again conducted by a member from this group - the details can be found in \cref{collab:sprintend:three}

\subsection{Backlog}

There are some general tasks to be carried over to the next sprint:

\begin{itemize}
\item The new profile selector from \textit{\giraf Components} needs to be implemented.
\item The code base should in general be cleaned up and refactored to ensure structural integrity of the application.
\item The loading animation when starting \launcher statically shows the logo for two seconds and should only be shown for the amount of time it takes to load the needed data.
\end{itemize}

The main task, however, is to refactor and improve the settings:

\begin{itemize}
\item The list view on the left in \settingsactivity needs to display the chosen tab correctly and proper shadowing should be shown.
\item Loading applications in both \lstinline!GirafFragment! and \lstinline!AndroidFragment! should display a loading animation to the user and load them in a background thread.
\item Installing a new application should update the available ones in \lstinline!GirafFragment! and \lstinline!AndroidFragment!.
\end{itemize}

This concludes the third sprint.