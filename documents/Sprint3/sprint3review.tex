\subsection{Evaluation}
Implementing the settings was the primary task for the third sprint.
The task consisted of several subtask, which have been completed:

\begin{itemize}
\item It is possible to scale applications up and down and disable the start up animation.
\item It is possible to add and remove both \giraf and Android applications on a 'per user' basis.
\item It is possible to access the Play Store and the native Android settings from settings.
\item It is possible to access the settings of other \giraf applications through \launcher 's settings.
\end{itemize}

This means the task has been solved and the sprint has been a success. 

We decided not to present our progress at this sprint review meeting. While it technically goes against Scrum practice not demonstrate the current state of the project to the clients at each sprint review, \settingsactivity in its current state is essentially identical to the prototypes previously presented to the clients. Furthermore, our clients do not have the same urge to monitor progress, as paying costumers might have.

Sprint end is once again conducted by a member from this group - the details can be found in \cref{collab:sprintend:three}

\subsection{Backlog}

There are some general tasks to be done:

\begin{itemize}
\item The new Profile Selector from \textit{GUI} needs to be implemented.
\item The code base should in general be cleaned up and refactored to ensure structural integrity of the application.
\item The loading animation in \lstinline!MainActivity! statically shows the logo for two seconds and should only be shown for the amount of time it takes to load the needed data.
\end{itemize}

The main task, however, is to refactored and improved the settings:

\begin{itemize}
\item The ListFragment on the left needs to display the chosen pane correctly and proper shadowing should be shown.
\item Loading applications in both \lstinline!GirafFragment! and \lstinline!AndroidFragment! should display a loading animation to the user and load them in a background thread.
\item Installing a new application should update the available ones in \lstinline!GirafFragment! and \lstinline!AndroidFragment!.
\end{itemize}