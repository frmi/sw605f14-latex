\subsection{Disabling the Drawer}
As described in \cref{sec:sprint2:firstmeeting,sec:sprint2:secondmeeting}, the purpose of the drawer is founded in a misinterpretation.
It was by the previous groups working with \launcher thought as being a requirement to be able to colour the each \giraf applications accessible from \launcher.
However, at the customer meetings, it was clarified no such requirements for the applications exist;
what the customers want is to switch the colours of \textit{pictograms}, being able to also show them in black and white, since some citizens handle coloured pictures better than black and white ones and vice versa, as referred to in \cref{sec:sprint2:clarification}.\\

Some additional problems are found with the drawer.
These are mostly centred upon other applications absorbing the colour given to them by \launcher, and using that as their main theme.
However, this means that for each possible colour the \launcher can give an application, it needs to implement a colour scheme that fits with the given colour.
This is especially a problem for applications using a wide variety of colours.\\

Since the purpose of the drawer is not a requirement per se, and the colouring functionality caused problems in other applications, it is decided to disable the drawer since it serves no particular purpose.
In effect, the code for animating the drawer remains as is, in case the functionality at a later time will prove useful.
This way, should a future group need to implement the drawer, they merely have to uncomment part of the code.