\subsection{Design of Settings Activity}\label{sec:sprint3:designsettings}
\thilemann{The settings activity should be better introduced - we have not yet talked about it, so talking about settings activity makes not sense! Should be an extension of the prototypes we made}
Here, the organization of the \lstinline!SettingsActivity! is described.
This includes how the list to the left is made and how the addition of apps to a user is done. \vagner{REMEMBER that this is ONLY sprint 3, so we only describe the basics here! Also, remember to add a nice figure, showing the structure.}



% THIS FILE SHOULD BE DELETED AND REMOVED FROM THE INDEX
% IT NO LONGER SERVES ANY PURPOSE!!!
\subsection{DELETE ME!!!}

\frederik[inline]{Jeg ved ikke hvor foelgende hoerer til}
\thilemann[inline]{Jeg tror det var da alle views med ikoner blev konverteret til bitmaps for bedre at skalere ikoner?}
{\color{red}* | ed2d9a5 (3 weeks ago) fmikke11@student.aau.dk changed views to imageViews. Apps are now shown as Bitmaps}


\subsubsection{Loading Apps into View}\label{sec:sprint:designlauncher}
The methods that load applications into a given view are in \lstinline!HomeActivity!.
In \lstinline!SettingsActivity! the pane where applications are added and removed from users also needs to load applications into a given view.
Therefore, we refactor \lstinline!HomeActivity! and move the relevant functions into \lstinline!LauncherUtility!.
Then both activities can call the same method and code redundancy is reduced. 

* | ed2d9a5 (3 weeks ago) fmikke11@student.aau.dk changed views to imageViews. Apps are now shown as Bitmaps\\ 

\subsubsection{Loading Apps into View}\label{sec:sprint:designlauncher}

The methods that load applications into a given view are in \lstinline!HomeActivity!.
In \lstinline!SettingsActivity! the pane where applications are added and removed from users also needs to load applications into a given view.
Therefore, we refactor \lstinline!HomeActivity! and move the relevant functions into \lstinline!LauncherUtility!.
Then both activities can call the same method and code redundancy is reduced. 