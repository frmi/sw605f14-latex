\subsection{Design of Settings Activity}\label{sec:sprint3:designsettings}
\thilemann{The settings activity should be better introduced - we have not yet talked about it, so talking about settings activity makes not sense! Should be an extension of the prototypes we made}
Here, the organization of the \lstinline!SettingsActivity! is described.
This includes how the list to the left is made and how the addition of apps to a user is done. \vagner{REMEMBER that this is ONLY sprint 3, so we only describe the basics here! Also, remember to add a nice figure, showing the structure.}

\subsection{The Settings Library}
The idea behind the settings library was to create a streamlined and uniform user interface for changing settings across the entire \giraf project. As discussed in \jesper{Indsæt ref til kundemøde 2}, we agreed with the clients that settings should accessible through two methods:
\begin{itemize}
	\item A dialogue box in each individual application, containing only settings relevant to that specific application. The dialogue box should be displayed when the user presses a button that is identical throughout all \giraf applications, making it easier for users to find.
	\item A unified settings activity, where the user has immediate access to all settings for all \giraf applications. This activity should be a part of the Launcher. Besides the settings of the individual applications, it should also contain relevant settings pertaining to the user, e.g. which applications the user should be allowed access to.
\end{itemize}

It is immediately apparent that for making both these approaches work concurrently, we would need to create a single plan for how user settings should be handled throughout the \giraf project. In the following we will describe this plan.

\subsubsection{Concept}
We decided that the entire system should consist of four components, to be implemented separately:
\begin{itemize}
	\item The unified settings activity should be implemented as a part of the Launcher, and therefore the implementation of this component naturally falls to ourselves. Excluded from this task is the settings of the individual applications.
	\item A dialogue box able to contain the settings for an arbitrary \giraf application. Included in this is a standard button for launching the dialogue box. As this is user interface component useful to all applications, it should be implemented by the group responsible for the \giraf GUI library.
	\item A set of settings for each \giraf applications. As these sets may vary widely in content and complexity from application to application, and furthermore may change along with their associated applications over time, these should be defined and implemented by the individual project groups. 
	\item A method for storing the settings in the database, allowing the users to easily use the same settings across several devices. This should be implemented by the database group. 
\end{itemize}

\subsubsection{Technical Design}
The big challenge of implementing the above concept, is how to allow the settings interface layout of each application to also be used in the unified settings activity. 

Our initial idea was to ask each group to create a fragment with their settings layout within their own activity, which our settings activity would then be able to load into the unified settings activity. This would isolate the responsibilities of each group, and keep overall complexity to a minimum. However, after some research it turns out that sharing fragments across applications is an ``unnatural'' operation in Android\frederik{Readthrough make real explanations find source.}. While possible, it is not very elegant, even when only sharing a single fragment between two applications, and it results in considerable complexity when scaled to our needs.\frederik{Hvordan var det præcis det virkede?} To simplify this, \launcher could include all other projects as dependencies when compiling, but this would have made the \launcher project unnecessary bloated and difficult to work with.

Another possibility was make the individual applications implement an interface, where a function would add the application's setting-widgets to a given view. While Android provides facilities for inter-application communication through its \textit{Broadcast} mechanism\jesper{indsæt cite}, this requires all the involved applications to be running. This is not an acceptable condition in this situation. 

We finally decided on creating a common settings library, to which all groups added a layout fragment for their own application. This library should then be included with every project that contributes to it, so they can read their own settings from it, and to \launcher, which can use the layouts for the settings activity. This requires a single project to which all developers on the \giraf project have write access, which makes it hard to manage, but it is on the other hand the simplest technical solution. A concrete disadvantage is that errors in the various settings fragments will influence \launcher, but our group will have complete authority over the settings library project, and have the right to make changes necessary to allow \launcher to compile and execute. 


* | ed2d9a5 (3 weeks ago) fmikke11@student.aau.dk changed views to imageViews. Apps are now shown as Bitmaps\\ 

\subsubsection{Loading Apps into View}\label{sec:sprint:designlauncher}

The methods that load applications into a given view are in \lstinline!HomeActivity!.
In \lstinline!SettingsActivity! the pane where applications are added and removed from users also needs to load applications into a given view.
Therefore, we refactor \lstinline!HomeActivity! and move the relevant functions into \lstinline!LauncherUtility!.
Then both activities can call the same method and code redundancy is reduced. 