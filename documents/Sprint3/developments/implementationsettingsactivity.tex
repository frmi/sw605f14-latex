\subsection{SettingsActivity}
The class \lstinline|SettingsActivity| is responsible for \textit{inflating} the settings layout.
The layout inflated is seen in \cref{lst:settingsactivity:layout}.
Its layout is fairly simple, since it is not responsible for rendering any contents.

\begin{lstlisting}[caption={Excerpt of the layout defined for \lstinline|SettingsActivity|.}, label={lst:settingsactivity:layout}]
<fragment
  android:id="@+id/settingsListFragment"
  android:layout_weight="2"
  class="dk.aau.cs.giraf.launcher.settings.SettingsListFragment"
</fragment>

<FrameLayout xmlns:android="http://schemas.android.com/apk/res/android"
  android:id="@+id/settingsContainer"
  android:layout_weight="6">
</FrameLayout>
\end{lstlisting}

Instead it defines two containers.
The first being the \lstinline|<fragment>| element, which is the Android way of defining a \textit{static} fragment that can not be changed after instantiation.
As indicated by the \lstinline|class| attribute, it is inflating the \lstinline|SettingsListFragment|, which purpose is explained in \cref{sec:settingslistfragment}.
The second container being a \lstinline|FrameLayout| that will contain the settings selected from \lstinline|SettingsListFragment|.

Since Android disproves of inter-fragment communication, \lstinline|SettingsActivity| will have to stand between and respond to requests send from its active fragments.
This is done by implementing the \lstinline|SettingsListFragment.SettingsListFragmentListener| interface to define methods for what the \lstinline|SettingsListFragment| wants to communicate, as seen in \cref{lst:settingslistfragment:interface}.
Defining the methods of the interface, no coupling between the fragment and activity exist because the fragment does not know of the activity.

\begin{lstlisting}[caption={The interface implemented in \lstinline|SettingsActivity| defined in \lstinline|SettingsListFragment|.}, label={lst:settingslistfragment:interface}]
public interface SettingsListFragmentListener {
  public void setActiveFragment(Fragment fragment);
  public void reloadActivity();
  public ArrayList<SettingsListItem> getInstalledSettingsApps();
  public void setCurrentUser(Profile profile);
}
\end{lstlisting}


\subsection{SettingsListFragment}\label{sec:settingslistfragment}
This is the static fragment defined in the \lstinline|SettingsActivity| layout (\cref{lst:settingsactivity:layout}).



\subsubsection{SettingsListItem}
\lstinline|SettingsListItem| is a simple class, which members are used to render the information shown in the \lstinline|ListView|.
It is thus used as a placeholder for the fragments and intents that should be opened by clicking an list item.
The members of this class are seen in \cref{lst:settingslistitem}.

\begin{lstlisting}[caption={Members of the \lstinline|SettingsListItem| class.}, label={lst:settingslistitem}]
String mPackageName;
String mAppName;
String mSummary;
Drawable mAppIcon;
Fragment mAppFragment;
Intent mIntent;
\end{lstlisting}

By using constructor chaining, the class provides various public constructors that use one private constructor to set the values of its members.
This is decided since the \lstinline|ListView| in \lstinline|SettingsListFragment| have to be able open both fragments and intents.
Otherwise it will not be possible to open settings for other \giraf applications.
Logic used in \lstinline|SettingsActivity| checks if either the fragment or intent member is \lstinline|null|, and starts a new activity accordingly, when an item is clicked.

\subsubsection{Adding non-launcher Applications to Settings}


\subsection{SettingsListAdapter}

\subsection{Saving Settings}

\subsection{Settings Specific to \launcher}

\subsubsection{Launcher Settings}
This section describes the pane which contains the settings for the \launcher, as described in \Cref{sec:sprint3:designsettings}.

Before describing the settings and implementation of the settings in this pane we explain how settings can be managed in Android.

\paragraph{Managing settings in Android}\label{para:sprint4:managingsettingsandroid}
To explain the functionality we use to save preferences in our application this section describes the excellent features which Android provides for managing preferences on a per application basis. This feature is provided through the SharedPreference class.

With this class you can either save the preferences in the context of the application, if they are application specific. It is also possible to define the name of the file in which the preferences are saved or read from. If the preference file does not exist when requested it is created.

We use these features to save preferences for each user. We create a unique file by the user role and id.
This means that a \textit{citizen} profile with the \textit{id 1}, would get the file name \textit{c1}, with the package name of our application as a prefix.

Preferences identified by a key, and they are edited through the Editor class and with a proper method matching the type of the variable which needs to saved, e.g. \lstinline!putBoolean()!.
If the preference exists it is updated if not it is created.

When a preference is to be read it is retrieved by a get method, e.g. \lstinline!getBoolean()!, this method is provided the key for the preference and a default value in case the preference has not yet been set.

\paragraph{The settings} implemented in the GIRAF pane are
\begin{itemize}
 	\item disabling of the start animation, when starting \launcher the first time after a system restart,
 	\item and changing the size of application icons on the home activity.
 \end{itemize} 
It should be noted that these settings are added as suggestions to which settings could be relevant and they made it to the final release.

The most interesting of the two preferences is the icon scaling preference, since the ``Show start up animation" preference is created with a standard Android preference component which provides the UI with a two-state switch.

The icon scaling preference is created by extending the standard preference class in Android there by inheriting the native setup, e.g. title and summery fields, and constructors for such a component. We have then supplied it with a custom layout containing a slider and an application icon as those used in the home activity.

Settings in this pane are automatically saved when the user either changes pane or exits the activity.


\subsubsection{Application Settings}
\subsection{Implementation of AppContainerFragment}
This section will describe the \lstinline!AppContainerFragment! and how it loads \lstinline!GirafFragment! and  \lstinline!AndroidFragment!.
Once again, remember that this is Sprint 3 and should therefore only be the things included in sprint 3!