This is the static fragment defined in the \lstinline|SettingsActivity| layout (\cref{lst:settingsactivity:layout}).

The main tasks of \lstinline|SettingsListFragment| is to update \lstinline|SettingsActivity| using the interface callback member \lstinline|private SettingsListFragmentListener mCallback|.
This is done in the appropriate listeners for the \textit{Profile Selector} and \textit{ListView} as illustrated in \cref{lst:settingsactivity:layout}.
In the implementation, these components corresponds to a \lstinline|GProfileSelector| and a standard Android \lstinline|ListView| respectively.

To ensure that the context of the underlying activity implements the interface just described, the code in \cref{lst:settingslistfragment:onattach} is utilized in \lstinline|SettingsListFragment|.

\begin{lstlisting}[caption={Implementation to make sure the underlying activity implements the \lstinline|SettingsListFragmentListener| interface.}, label={lst:settingslistfragment:onattach}]
@Override
public void onAttach(Activity activity) {
  super.onAttach(activity);
  try {
    mCallback = (SettingsListFragmentListener) activity;
  } catch (ClassCastException e) {
    throw new ClassCastException(activity.toString()
            + " must implement SettingsListFragmentListener");
  }
}
\end{lstlisting}

To enable its global member \lstinline|mSettingsListView| to render its items, an adapter (\lstinline|mAdapter|) is instantiated with a list of installed applications, before being activated for the \lstinline|ListView| by calling \lstinline|mSettingsListView.setAdapter(mAdapter)|.

Before describing this adapter in \cref{sec:settingslistadapter}, the meaning of \lstinline|SettingsListItem| is explained.


\subsubsection{Content of the Settings Menu}
\lstinline|SettingsListItem| is a simple class, whose members are used to render the information shown in the \lstinline|ListView|.
It is thus used as a place holder for the fragments and intents that should be opened by clicking an list item.
The members of this class are seen in \cref{lst:settingslistitem}.

\begin{lstlisting}[caption={Members of the \lstinline|SettingsListItem| class.}, label={lst:settingslistitem}]
String mPackageName;
String mAppName;
String mSummary;
Drawable mAppIcon;
Fragment mAppFragment;
Intent mIntent;
\end{lstlisting}

By using constructor chaining, the class provides various public constructors that use one private constructor to set the values of its members.
Depending on the constructor used, for each item of the list view in \lstinline|SettingsListFragment|, the item is able to either change the fragment in the \textit{Settings Container} or start an activity through an intent.
The fragment is changed using the \lstinline|FragmentManager| which is used to manage fragment transactions.
The list view should also be able to contain an item for settings in other \giraf applications, therefore the necessity for launching activities through intents.
Logic used in \lstinline|SettingsActivity| checks if either the fragment or intent member is \lstinline|null|, and starts a new activity accordingly, when an item is clicked.
