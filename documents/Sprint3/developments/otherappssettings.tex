\subsection{Adding to Settings}
The interface implemented by \lstinline|SettingsActivity| forces it to implement a method called \lstinline|getInstalledSettingsApps()| that has to return a list consisting of \launcher settings, intents available from other \giraf applications and intents related to the native Android settings.
The method returns an \lstinline|ArrayList<SettingsListItem>| by utilizing a range of methods that all adds an item to the list, whether it be a fragment or intent.
This means that each \giraf application that wants to be available through \settingsactivity in \launcher, has to be added to \lstinline|getInstalledSettingsApps()|.

To enable another \giraf application to be shown in settings, it must provide the following intent filter declared for the activity containing its settings, as illustrated in \cref{lst:intentfilterxml}.

\begin{lstlisting}[caption={The intent filter and action a \giraf application has to provide to be shown in settings.}, label={lst:intentfilterxml}]
<intent-filter>
  <action android:name="dk.aau.cs.giraf.cars.SETTINGSACTIVITY"/>
  <category android:name="android.intent.category.DEFAULT"/>
</intent-filter>
\end{lstlisting}

Important to notice is that the name of the provided action within the intent filter, is a key having the value \lstinline|SETTINGSACTIVITY|, which is appended to the full package name of the application to show settings for.