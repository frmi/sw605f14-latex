\thilemann[inline]{As mentioned a bit earlier, the part where preferences are mentioned in loadapplications(?) should be placed after this section...}
This section explains the functionality that saves preferences when a change is made by describing the features, which Android provides for managing preferences on a per application basis, as well as how we use it.

Android provides the class \lstinline|SharedPreferences|, which is used to handle settings such as ours.
This class provides functionality to save preferences in the context of an application by writing to a file on the device.

We must create the files through \lstinline|SharedPreferences| manually.
Creating a unique file name is achieved by combining the role and id of the current user.
This means that a \textit{citizen} profile with the id \textit{1}, would get the file name \textit{c1}, prepended with the package name of our application.

Preferences are identified by a key, and edited through the \lstinline|Editor| class, which provide methods to edit preferences.
An example of one of the messages is the \lstinline!putBoolean()! which stores boolean field in a preference, identified with a provided key.
If the preference exists it is updated - if does not, it is created.

When a preference is to be read, it is retrieved by a \lstinline|get()| method, e.g. \lstinline!getBoolean()!.
This method is provided the key for the preference and a default value in case it has not yet been set.\\

In our application, we are managing settings using the Android activity and fragment life cycles.
In the \settingsactivity we are reading and storing settings when each fragment is shown and closed respectively. 
By doing this the user will not have to do anything to load the preferences or store changes.