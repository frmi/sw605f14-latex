Considering the prototypes seen in \cref{fig:settingsprototype,fig:appsmanagement} it is apparent that a specially designed view (also referred to as an \lstinline|AdapterView|) for each list item is required.
In Android, the easiest way of doing so, is by implementing an \lstinline|Adapter| for the \lstinline|ListView|.
An adapter serves the objective of bridging the underlying data for a view, and to render it appropriately.
Various adapters exist for different purposes; it is e.g. possible to implement an extension of \lstinline|Adapter| called \lstinline|ListAdapter|, which serves the purpose of easily adding items to a \lstinline|ListView|.
Implementing an adapter also has performance related benefits.
Most importantly being the concept of \textit{recycling} a view, which makes it possible to handle large lists efficiently.\thilemann{Maybe add ref?}

In the case of yielding a visual result as close as possible to the prototypes, our implementation requires a custom class extending the \lstinline|BaseAdapter| class.
The specific reason being the list requiring both a title and an icon, but also specifically because of the shadows rendered around a selected item in the list.
This would not be possible to handle with a standard adapter.

The most important method of an adapter is its \lstinline|getView()| method, since it is solely responsible for returning a new \lstinline|AdapterView| to the underlying \lstinline|ListView|.
This method is seen in \cref{lst:settingslistadapter:getview}.

\begin{lstlisting}[caption={\lstinline|getView()| method in \lstinline|SettingsListAdapter| class.}, label={lst:settingslistadapter:getview}]
public View getView(int position, View convertView, ViewGroup parent) {
  View vi = convertView;
  
  if(convertView == null)
    vi = mInflater.inflate(R.layout.settings_fragment_list_row, null);
  
  // Get current item in the list
  SettingsListItem item = mApplicationList.get(position);
  
  if (position == mLastSelectedItem) {
    setShadowRightCurrent(vi, false);
    ListView listView = (ListView)parent.findViewById(R.id.settingsListView);
    listView.setItemChecked(position, true);
  }
  else {
    setShadowRightCurrent(vi, true);
  }
  
  if (position == mLastSelectedItem - 1)
    setShadowAboveCurrent(vi, true);
  else
    setShadowAboveCurrent(vi, false);
  
  if (position == mLastSelectedItem + 1)
    setShadowBelowCurrent(vi, true);
  else
    setShadowBelowCurrent(vi, false);
  
  ImageView appIcon = (ImageView)vi.findViewById(R.id.settingsListAppLogo);
  TextView appNameText = (TextView)vi.findViewById(R.id.settingsListAppName);
  
  // Setting all values in ListView
  appIcon.setBackgroundDrawable(item.mAppIcon);
  appNameText.setText(item.mAppName);
  
  return vi;
}
\end{lstlisting}

In the \lstinline|getView()| method the most important parameters are \lstinline|position| and \lstinline|convertView|.
\lstinline|position| being the index of the item in the \lstinline|ListView| that is currently being inflated.
\lstinline|convertView| is the view obtained from the \lstinline|ListView|.
As seen in the example, this view is only inflated if it is not \lstinline|null|, which is one of the optimizations of using an adapter.
\thilemann[inline]{Check what this has to do with recycling}

To be able to add data to each \lstinline|View| contained in a \lstinline|convertView|, the item related to the position is obtained, where a \lstinline|View| fx is the \lstinline|ImageView| showing the application icon.

The rest of the \lstinline|getView()| method is spend on checking logic for which shadows to show, based on the last selected item.
This logic utilizes three different methods looking much alike.
One of these are found in \cref{lst:settingslistadapter:shadow}

\begin{lstlisting}[caption={One of three methods setting the visibility of shadows related to the selected list item.}, label={lst:settingslistadapter:shadow}]
private void setShadowRightCurrent(View currentView, boolean visible) {
  View rightShadow = currentView.findViewById(R.id.settingsListRowShadowRight);
  if (visible)
    rightShadow.setVisibility(View.VISIBLE);
  else
    rightShadow.setVisibility(View.GONE);
}
\end{lstlisting}

This implementation is not considered the ``most optimal'', since it brings some code duplication for methods doing the same kind of task.
The naming is also not considered optimal, since it can be confusing which shadow is currently being set as visible.
The reason for doing this is that we need a reference for a specific view depending on the shadow needed to show.

\lstinline|mLastSelectedItem| is a global member in \lstinline|SettingsListAdapter|, made available through a public method called \lstinline|setSelected(int position)|.
This is called from an item click listener, implemented by the underlying \lstinline|ListView|.

\thilemann[inline]{Should we say comments are omitted in all code examples, or that this was done late during spring 4 to support the next group? Vagner: Jeg har nævnt at kommentarer er exkluderet alle steder hvor jeg har exkluderet dem.}
