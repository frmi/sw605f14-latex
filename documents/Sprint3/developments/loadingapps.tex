\subsection{Loading Applications}\label{sect:sprint3:refactoring}
In the new \lstinline!SettingsActivity!, described in \cref{sec:sprint3:settings}, \launcher will load applications similar to the way done in \lstinline|HomeActivity|.
However, the methods that load applications into a given view are placed in \lstinline!HomeActivity!.
Therefore, it is needed to refactor much of the existing code, so \launcher can load applications from any given activity.\\

\lstinline!HomeActivity! is refactored by moving the relevant methods, most prominently the \lstinline|loadApplications()| method, into \lstinline!LauncherUtility!.
Then both activities can call the same methods, and code redundancy is reduced. 
However, \lstinline|loadApplications()| needs to be changed further to accommodate the different usages in the \homeactivity and the \settingsactivity. 
It is required to pass different \lstinline|OnClickListner|s, since the behaviour is not the same.
The application management fragment requires the applications to be marked when pressed, while the home activity requires the applications to start.

\subsubsection{Observing New Applications}\label{sec:sprint3:observing}
To increase usability when installing or removing applications on the device, an observer which is responsible for keeping an eye on currently installed applications and compare these with those which are shown at that moment. 
If there fx are new applications on the device, which are not in the list of currently loaded applications, they are reloaded, by calling the \lstinline|loadApplications()| method described above.

This functionality is implemented creating a custom timer task by extending the Java class \lstinline!TimerTask! and scheduling this custom task to run at a fixed interval. 
Thus, a check is made every fifth second to see if any new applications have been installed.
The interval is decided after what we found acceptable by trial and error.

For every execution of the custom task, a list of installed applications is retrieved and compared with a list of all currently shown applications.
If the two lists differ, the layout is refilled and the list of currently shown applications is updated.