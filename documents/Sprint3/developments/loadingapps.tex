\subsection{LauncherUtility and Loading Apps}\label{sect:sprint3:refactoring}
In order to implement the \lstinline!SettingsActivity! much of the existing code needed to be refactored.
This was especially due to the need of loading Android applications that are not part of the GIRAF framework \vagner{framework? Project?} into the Launcher. 
Most of the functions from \lstinline|HomeActivity| that dealt with populating a view with applications are moved to \lstinline|LauncherUtility|.\vagner{It feels like I have read or written this before, in a different part of the report...}

\subsubsection{Refactoring of LauncherUtility}
The methods that load applications into a given view are in \lstinline!HomeActivity!.
In \lstinline!SettingsActivity! the pane where applications are added and removed from users also needs to load applications into a given view.
Therefore, we refactor \lstinline!HomeActivity! and move the relevant functions into \lstinline!LauncherUtility!.
Then both activities can call the same method and code redundancy is reduced. 

 ed2d9a5 (3 weeks ago) fmikke11@student.aau.dk changed views to imageViews. Apps are now shown as Bitmaps
* | | 01e1fae (13 days ago) thilemann@gmail.com Refactored how items are added to the list of applications. They are now added through either addApplicationByName or addApplicationByPackageName. addApplicati
onByPackageName is always preferred over the other, since it application name and icon are added dynamically by the packagename supplied. addApplicationByName is used when adding settings not part of a packa
ge, such as the one added for Android.\\
* | 8e6bc6f (3 weeks ago) fmikke11@student.aau.dk Refactored HomeActivity - Moved loadApplications from HomeActivity to LauncherUtility to make it accessable for other classes\\
* | 56bdf59 (2 weeks ago) fmikke11@student.aau.dk Refactored shared functionality in functions creating an app view/imageView\\

\subsubsection{Observing New Applications}\label{sec:sprint3:observing}
To increase usability when installing or removing applications on the device, we created an observer which has the responsible to keep an eye on current installed apps and compare these with those which are shown at that moment. If there for example are new applications on the device, which are not on the list of currently loaded apps they are reloaded.

This functionallity is implemented creating a custom timer task by extending the Java class \lstinline!TimerTask! and then scheduling this custom task to run at a fixed rate. The rate at which we check for new applications is every fifth second, this was the amount of time we found acceptable.

Every fifth second our custom task is retrieving a list of installed applications on the device and is comparing this list with the current loaded applications. If the two lists differ the applications are reloaded.