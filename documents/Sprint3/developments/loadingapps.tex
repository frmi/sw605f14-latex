\subsection{Loading Apps}\label{sect:sprint3:refactoring}
In order to implement the \lstinline!SettingsActivity! much of the existing code needed to be refactored.

The methods that load applications into a given view are in \lstinline!HomeActivity!.
In \lstinline!SettingsActivity!, the ``Apps'' pane also needs to load applications into a given view.
Therefore, we refactor \lstinline!HomeActivity! and move the relevant methods, most prominently the \lstinline|loadApplications()| method, into \lstinline!LauncherUtility!.
Then both activities can call the same method and code redundancy is reduced. 
However, \lstinline|loadApplications()| needed to be changed further, to accommodate loading Android applications that are not part of the GIRAF framework \vagner{framework? Project?} into \launcher. 
More specifically, the method was overloaded and then converted different type of lists of applications into the \giraf type \lstinline|List<AppInfo>|.\\

Furthermore, the code was in several case supplemented with additional functionality.

\subsubsection{Observing New Applications}\label{sec:sprint3:observing}
To increase usability when installing or removing applications on the device, we created an observer which has the responsible to keep an eye on current installed apps and compare these with those which are shown at that moment. If there for example are new applications on the device, which are not on the list of currently loaded apps they are reloaded.

This functionality is implemented creating a custom timer task by extending the Java class \lstinline!TimerTask! and then scheduling this custom task to run at a fixed rate. The rate at which we check for new applications is every fifth second, this was the amount of time we found acceptable.

Every fifth second our custom task is retrieving a list of installed applications on the device and is comparing this list with the current loaded applications. If the two lists differ the applications are reloaded.