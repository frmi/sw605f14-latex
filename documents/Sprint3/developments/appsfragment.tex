\subsubsection{Application Settings}
This section follows up from the design described in \cref{sec:sprint3:designsettings}.
When entering the application management fragment, the \lstinline!AppManagementFragment! is loaded into the settings container, seen in \cref{fig:settingsarchitecture}.
The application management fragment in turn contains another container, where the fragments showing the applications are loaded into.
Furthermore, \lstinline!AppManagementFragment! implements three buttons in the top of the layout: the \textit{Giraf}, the \textit{Android}, and the \textit{Butik} buttons.\footnote{``Butik'' is Danish for a ``store''.}\\

The \textit{Butik} button opens Google Play, so the user can download additional Android applications, if desired. 
We open the Play Store as a separate activity with certain flags, ensuring that the user returns to \launcher when exiting the Google Play.
The exact code can be seen in \cref{lst:launchergoogleplay}.\\

The \textit{Giraf} and the \textit{Android} buttons load a new fragment into  \lstinline!AppsContainer!, by using a \lstinline!FragmentManager! - the same way it is described in \cref{sec:settingslistfragment}.
The fragments loaded are derived from the same superclass, called \lstinline|AppContainerFragment| and is described subsequently.

\subsubsection{AppContainerFragment and its Derived Classes}

Because the \giraf and Android fragments contain many of the same variables, these inherit all shared information from a superclass, \lstinline!AppContainerFragment!, consequently reducing redundancy and clarifying how the two fragments are different.
The main responsibilities of the superclass are to initialize shared variables and implement shared methods that handle loading applications.

As a result, the two derived classes get their shared variables initialized by a call to \lstinline!super.onCreate()! and reloading of applications is handled by calls to \lstinline!AppContain!-\lstinline!erFragment!. \\

However, which types of applications are loaded are different for \lstinline!AndroidFragment! and \lstinline!GirafFragment!.
This is solved by initializing the shared variable \lstinline!apps! differently in the two derived classes -
\lstinline!GirafFragment! loads \giraf applications into \lstinline!apps!, while \lstinline!AndroidFragment! loads Android applications into it.

Since the automatic load methods mentioned above work on the \lstinline!apps! variable, this solution solves the problem. \\

The other problem is related to the marking of applications.
This is due to the fact that \giraf applications are saved as the \textit{OasisLib} type \lstinline!Application!, while Android applications are saved as the type \lstinline!ResolveInfo!.

The \giraf applications connected to a user are saved or removed through a call to an \textit{OasisLib} method, while Android applications are stored using the preferences system in Android.
More about saving settings in Android can be found in \cref{para:sprint4:managingsettingsandroid} and the code for marking the applications can be seen in \cref{appendix:markingapps}\\

These two problems are the main reasons we need to derive the two subclasses.\\

Having described all the different types of settings \launcher treats, the next sections explains how to save the settings.