\subsubsection{Application Settings}
\style{This looks a little messy with the bold fonts (and also inconsistent way of using it)}
This section follows up from the design described in \cref{sec:sprint3:designsettings}.
When clicking the ''Apps'' pane, the fragment loaded into the \textit{Settings Container} in \lstinline!SettingsActivity! is \lstinline!AppManagementFragment!.
\thilemann[inline]{Couldnt we instead write something like: ``when clicking/activating application settings, its fragment is...''?}
\lstinline!AppManagementFragment! in turn contains another fragment container called \lstinline!AppsContainer!.\thilemann{It is not described what this is - it is not relevant to know what it contains, but rather what it does.}
% where the two fragments derived from \lstinline!AppContainerFragment! will be loaded in - the fragments showing the applications.
Furthermore, \lstinline!AppContainerFragment! implements three buttons in the top of the layout: the \textbf{Giraf} button, the \textbf{Android} button and the \textbf{''Butik''} button.\footnote{''Butik'' is the danish word for ''store''}\\

%\begin{itemize}
%\item The \textbf{Giraf} button loads the \lstinline!GirafFragment! into \lstinline!AppsContainer!
%\item The \textbf{Android} button loads the \lstinline!AndroidFragment! into the \lstinline!AppsContainer!
%\item The \textbf{''Buik''} (Store) button opens the Play Store App. If the apps is not installed on the device, it opens the Play Store in the default browser.
%\end{itemize}

The \textbf{''Butik''} button opens the  Play Store, so the user can download additional Android applications, if desired. 
We open the Play Store as a separate activity with the \lstinline!Intent.FLAG_ACTIVITY_NEW_TASK! and the \lstinline!Intent.FLAG_ACTIVITY_CLEAR_TOP! flags.
The exact code can be seen in \cref{lst:launchergoogleplay}.\\

The \textbf{Giraf} and the \textbf{Android} buttons load a new fragment into \lstinline!AppsContainer!, by using a \lstinline!FragmentManager! - the same way it is described in \cref{sec:settingslistfragment}.\\

The fragments loaded are derived from the same superclass, called \lstinline|AppContainerFragment| and is described subsequently.
%An  \lstinline!OnClickListener! is attached to the \textbf{Giraf} and \textbf{Android} buttons, which call the method \lstinline!replaceFragment()!, sending the appropriate fragment as argument to \lstinline!replaceFragment()!.
%The method \lstinline!replaceFragment()! can be seen in \cref{lst:replaceFragment}.
%
%\begin{lstlisting}[caption={Method used to replace the fragment currently loaded into the fragment container in AppManagementFragment}, label={lst:replaceFragment}]
%/**
% * Replace active fragment by running the transaction in a new thread.
% * Adds responsiveness when loading list of installed apps_container.
% * @param fragment
% */
%private void replaceFragment(final Fragment fragment){
%    new Runnable() {
%        @Override
%        public void run() {
%            FragmentTransaction ft = fragmentManager.beginTransaction();
%            ft.replace(R.id.app_settings_fragmentlayout, fragment);
%            ft.commit();
%        }
%    }.run();
%}
%\end{lstlisting}

%Firstly, we researched how to use the inbuilt API for Google Services  to open the Play Store in this way.
%However, there were some problems with the API when attempting to implement.
%Furthermore, it was discovered that the API is intended to be used for syncronization with Google+, Google Drive and Google Games.


%The full design can be seen visualized in \cref{fig:settingsappfragments}.
 
%\begin{figure}[h]
%\centering
%\includegraphics[width=\textwidth, height=3in, keepaspectratio=true] {SettingsActivity.png}
%\caption{The organization of \lstinline!SettingsActivity! when inside the "Apps" pane. Since we need to distinguish between \giraf and Android applications, nested fragment containers are used.}
%\label{fig:settingsappfragments}
%\end{figure}

\subsubsection{AppContainerFragment and the derived classes}

Because the \giraf and Android fragments contain many of the same variables, these inherit all shared information from a superclass, consequently reducing redundancy and clarifying how the two fragments are different.
This superclass is called \lstinline!AppManagementFragment! and the main responsibilities are:

\begin{itemize}
\item Initialize shared variable in \lstinline!onCreate()!
\item Implement shared methods handling when to load applications
\end{itemize}

As a result, the two derived classes get their shared variables initialized by a call to \lstinline!super.onCreate()! and reloading of applications is handled automatically by \lstinline!AppContainerFragment!. \\

However, which type of applications are loaded is different for \lstinline!AndroidFragment! and \lstinline!GirafFragment!.
%As noted in \cref{sec:sprint:designlauncher} and explained in \cref{sect:sprint3:refactoring}, the functions used to load applications into view was moved from \lstinline!HomeActivity! to \lstinline!LauncherUtility!.
This is solved by initializing the shared variable \lstinline!apps! differently in the two derived classes -
\lstinline!GirafFragment! loads \giraf applications into \lstinline!apps!, while \lstinline!AndroidFragment! loads Android applications into it.

Since the automatic load methods mentioned above work on the \lstinline!apps! variable, this solution solves the problem. \\

The other problem present is related to the marking of applications.
%However, one problem has to be directly inside the two derived classes, namely marking the applications as selected and adding them to the users list of selected applications.
This is due to the fact that \giraf applications are saved as the \textit{OasisLib} type \lstinline!Application!, while Android applications are saved as the type \lstinline!ResolveInfo!.

The \giraf applications connected to a user are saved or removed through a call to an \textit{OasisLib} method, while Android applications are saved as a file stored in the native \lstinline!SharedPreferences! of Android.
%Each file name is made, unique to each user, by incorporating the users ID as part of the file name.
More about saving settings in Android can be found in \cref{para:sprint4:managingsettingsandroid} and the code for marking the applications can be seen in \cref{lappendix:markingapps}\\

%This also means that the two fragments must handle the selected applications different, when marking them the first time a fragment is loaded.
%The code is not included, since it is similar to that of \cref{lst:addinggirafapplications} and \cref{lst:addingandroidapplications} - the same checks are carried out, but rather than being for a single application, they are carried out on all installed applications.

These two problems are the main reason we need to derive the two subclasses.\\

Having described all the different types of settings \launcher treats, the next sections explains how to save the settings.