\subsection{Implementation of "Apps" in SettingsActivity}
This section follows up from the design described in \cref{sprint3:design:apps}.
The fragment filled into the fragment container in \lstinline!SettingsActivity! was named \lstinline!AppManagementFragment!.
\lstinline!AppManagementFragment! in turn contains another fragment container, \lstinline!AppsContainer!, where the two fragments derived from \lstinline!AppContainerFragment! will be loaded in - the fragments showing the applications.
Furthermore, \lstinline!AppManagementFragment! implements three buttons in the top of the layout:

\begin{itemize}
\item The \textbf{GirafButton} loads the \lstinline!GirafFragment! into the fragment container
\item The \textbf{AndroidButton} loads the \lstinline!AndroidFragment! into the fragment container
\item The \textbf{GooglePlayButton} opens the Play Store App. If the apps is not installed on the device, it opens the Play Store in the default browser.
\end{itemize}

While the \textbf{GirafButton} and the \textbf{AndroidButton} simply load a new fragment into \lstinline!AppsContainer!, the code behind the \textbf{GooglePlayButton} is more interesting and launches the correct intent. \vagner{finish writting this section and also, should we include code for the intent launching}

\subsubsection{AppManagementFragment}



 and how it loads \lstinline!GirafFragment! and  \lstinline!AndroidFragment!.

\vagner{Once again, remember that this is Sprint 3 and should therefore only be the things included in sprint 3!}