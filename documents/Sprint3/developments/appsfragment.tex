\subsection{Implementation of "Apps" in SettingsActivity}
This section follows up from the design described in \cref{sprint3:design:apps}.
The fragment filled into the fragment container in \lstinline!SettingsActivity! was named \lstinline!AppManagementFragment!.
\lstinline!AppManagementFragment! in turn contains another fragment container, \lstinline!AppsContainer!, where the two fragments derived from \lstinline!AppContainerFragment! will be loaded in - the fragments showing the applications.
Furthermore, \lstinline!AppManagementFragment! implements three buttons in the top of the layout:

\begin{itemize}
\item The \textbf{Giraf} buton loads the \lstinline!GirafFragment! into the fragment container
\item The \textbf{AndroidButton} loads the \lstinline!AndroidFragment! into the fragment container
\item The \textbf{Google Play} button opens the Play Store App. If the apps is not installed on the device, it opens the Play Store in the default browser.
\end{itemize}

The \textbf{Giraf} button and the \textbf{Android} button loads a new fragment into \lstinline!AppsContainer!.
The \textbf{Google Play} button launches the correct intent - The Play Store application or opening the native brower with the Play Store url.

\vagner{finish thissection}
\subsubsection{AppManagementFragment}

 and how it loads \lstinline!GirafFragment! and  \lstinline!AndroidFragment!.

\vagner{Once again, remember that this is Sprint 3 and should therefore only be the things included in sprint 3!}