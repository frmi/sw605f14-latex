\newcommand{\headerIntroduction}{Introduction}
\chapter*{\headerIntroduction}\label{chap:introduction}
\addcontentsline{toc}{chapter}{\headerIntroduction}
\jesper[inline]{Fix this section in relation to the introduction of the GIRAF chapter.}

The desire for the latest technology, whether it be a new car that enhances the safety of driving it, a gesture enabled television set or the constant craving for the latest mobile phones and tablets, often reflects a wish of improving certain areas of our lives using technology.
This report focuses on improving the way-of-life of people diagnosed with different kinds of \textit{Autism Spectrum Disorders} (ASD), including autism and Asperger syndrome.
The goal is to develop an Android software suite for tablet computers, consisting of tools that strives to replace some of the children's daily routines. This suite is called \giraf.
Example of \giraf applications include an interactive application that guides the children in putting on their outdoor clothes in the best order, and a game that focuses on improving the speech of speech-impaired children (referring to XXX\jesper{Add ref} and Cars respectively, which are included in \giraf).

\giraf is developed in cooperation with Aalborg Municipality, which has several institutions specializing in children and adults with various kinds of Autism Spectrum Disorders.
These institutions rely heavily on paper-based tools for their work with autistic children and adults. 
The idea behind \giraf is to organize and streamline these tools, by implementing them on mobile tablet devices.
These institutions are considered the \textit{customers} of \giraf, and therefore play a key role in guiding the developers, so \giraf can become as useful to them as possible.

The ideas of the customers are formalized by a group of students as a requirement specification, which contains both essential and inessential requirements for the further development of each application.

The origin of the \giraf project and a description of its most important components are described in \cref{chap:giraf}.
\thilemann{Include req.spec. in appendix and make reference here.}
These, together with our own contact with the customers, lays the ground for the work done in the sprints following the first sprint, which essentially focuses on getting to know \giraf and also fixing problems that were not fixed during the development cycles of the last groups working on it.