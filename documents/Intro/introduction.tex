\newcommand{\headerIntroduction}{Introduction}
\chapter*{\headerIntroduction}\label{chap:introduction}
\addcontentsline{toc}{chapter}{\headerIntroduction}

The desire for the latest technology, whether it be a new car that enhances the safety of driving it, a gesture enabled television set or the constant craving for the latest mobile phones and tablets, often reflects a wish of improving certain areas of our lives using technology.
This report focuses on improving the way-of-life of children diagnosed with different kinds of \textit{Autism Spectrum Disorders} (ASD), including autism and Asperger syndrome.
This is achieved by developing an Android software suite for tablets called \giraf, consisting of tools that strives to replace some of the children's daily routines.
It could for example be an interactive application that guides the children in putting on their outdoor clothes in the best order, or a game that focuses on improving the speech of speech-impaired children (referring to XXX and Cars respectively, which are included in \giraf).

\giraf is developed in collaboration with various institutions located in or close to Aalborg.
These institutions are considered the \textit{customers} of the multi-project, and are thus responsible for guiding the project in a desired direction, based on their individual ideas and needs for the applications in \giraf.


The ideas of the customers are formalized (by another group in the multi-project) as a requirement specification, which contains both essential and inessential requirements for the further development of each application.
\thilemann{Include req.spec. in appendix and make reference here.}
These, together with our own contact with the customers, lays the ground for the work done in the sprints following the first sprint, which essentially focuses on getting to know \giraf and also fixing problems that were not fixed during the development cycles of the last groups working on it.

\thilemann{Maybe write about Launcher - if working on it during all sprints.}