\newcommand{\headerIntroduction}{Introduction}
\chapter*{\headerIntroduction}\label{chap:introduction}
\addcontentsline{toc}{chapter}{\headerIntroduction}

The desire for the latest technology, whether it be a new car that enhances the safety of driving it, a gesture enabled television set or the constant craving for the latest mobile phones and tablets, often reflects a wish of improving certain areas of our lives using technology.
This report focuses on improving the way-of-life of people diagnosed with different kinds of \textit{Autism Spectrum Disorders} (ASD), including autism and Asperger syndrome.
The goal is to develop an Android software suite for tablet computers, consisting of tools that strive to replace some of the citizen's daily routines. This suite is called \giraf.
Examples of \giraf applications include \textit{Livshistorier}, an interactive application that guides the citizen in putting on their outdoor clothes in the best order, and \textit{Stemmespillet}, a game that focuses on training the voice of speech-impaired citizens.
A subset of the \giraf applications are described briefly in \cref{sec:giraf:applications:frontend}.

\giraf is developed in cooperation with Aalborg Municipality, which has several institutions specializing in children and adults with various kinds of ASDs.
These institutions rely heavily on paper-based tools for their work with autistic citizen and adults. 
The idea behind \giraf is to organize and streamline these tools, by implementing them on mobile tablet devices.
These institutions are considered the \textit{clients} of \giraf, and therefore play a key role in guiding the developers, so \giraf can become as useful to them as possible.

The ideas of the clients are formalized by a group of students as a requirements specification and user stories, which contains both essential and inessential requirements for the further development of each application, as described in \cref{appendix:requirements}.