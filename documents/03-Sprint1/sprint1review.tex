As we are using the Scrum method of development for the entire \giraf project, a sprint review meeting was planned for the end of the iteration. 
Larman \cite[p. 71]{larmanAgile} describes this type meeting as follows:
\begin{quote}
``At the end of each iteration, there is a review meeting (maximum of four hours)
hosted by the Scrum Master.
The team, Product Owner, and other stakeholders attend.
There is a demo of the product.

Feedback and brainstorming on future directions is encouraged, but no commitments are made during the meeting.
Later at the next Sprint Planning meeting, stakeholders and the team make commitments."
\end{quote}

At our meeting, the Scrum master of the \giraf project started with a short presentation, and then gave the floor to the groups\jesper{Describe the context of the GIRAF project early in the report.}, who spent five minutes each demonstrating their progress to the stakeholders\jesper{In aforementioned context description, define the stakeholders of the GIRAF project}. 
After the demonstrations, the stakeholders left, and groups proceeded to distribute the sub-projects among them, in preparation for the second sprint.

We do not consider this sprint review meeting a success, due to the following reasons:
\begin{itemize}
	\item It was a one-way meeting, where the stakeholders were presented with the progress, but were not encouraged to give any feedback. As a result, the development teams had no gain from the meeting, and were ill prepared for the second sprint. This problem was exacerbated by the lack of the sprint planning meetings required in Scrum to start a new sprint. As Larman \cite[p. 70]{larmanAgile} describes, during the sprint planning meetings goals are re-prioritized, and a new sprint backlog is created. 
	\item Several of the demonstrations had technical problems, such as applications crashing in the emulator used for the demonstration. This surprised the presenting groups, and thus they were unable to demonstrate their actual progress. A related problem was that several groups did not direct their presentation towards the costumers, and therefore focused on technical details, using a very technical vocabulary. The demonstration should be better prepared, with more focus on the actual goal of the meeting.
	\item Not all stakeholders were able to attend, as the meeting planned at a relative late date. As the main goal of the meeting is for the stakeholders review the progress of the project, this is not optimal.
\end{itemize}

These problems were discussed after the conclusion of the meeting. IT was suggested that a possible root problem was the \giraf Scrum master being overworked with planning this meeting along with his other responsibilities. It was decided to, at a later date, discuss the possibility of designate another person as responsible for conducting the sprint review meeting, and possibly the sprint planning meetings.

As a consequence of the lacking feedback, the \launcher sub-project had no clear goals for the second sprint, as most obvious issues had been resolved during the first sprint. We therefore decided get in contact with one of the costumers early in the second sprint, and discuss Launcher. We then expected to able to build a backlog for the second sprint along with this costumer. This is not an optimal solution, as the backlog should be compiled and prioritized as part of a coherent development process. But this solution will at least give us the ability to work from the requests of a costumer, rather than from what we ourselves think the costumers want. This costumer contact will be discussed further in \jesper{Ref til afsnit om kundekontakt, sprint 2.}


