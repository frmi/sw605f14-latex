\section{Testing}\label{sec:testing_sprint1}

As one of our goals was to increase the reliability of Launcher, we decided to conduct functional testing on the application.
We studied the specifications formulated by the earlier development teams in their 2011 \cite{launcher2011} and 2012 \cite{launcher2012} reports, and then studied how well Launcher corresponded to these specifications, by using the program in a number of different ways. 

Furthermore, a number of test cases were not based directly on the old specifications, but on what we found reasonable to expect of the application, based on our own knowledge of the system.

The tests described here were all performed by operating the running application, and were explicitly not based on our knowledge of the source code. 
This characterises our tests as dynamic black box testing. 

A number of test cases that would have been relevant, were not performed in this sprint, as they would have been too time consuming. 
These tests required significant changes to the test data built into the GIRAF system, or repeatedly installing and uninstalling applications from the device.
As there, at the time, was no central repository of stable application builds, most of the time would be spent building, installing and resetting different applications.

We only describe the results we found most interesting in this section, mainly tests that failed in a significant way. \jesper{Should we have a test appendix with the full results?}\vagner{I think so.}

\subsection{Specifications}
In the 2011 report by Andersen et al. \cite{launcher2011} the following specification for Launcher is presented:
\begin{quote}
\begin{itemize}
	\item \textit{List only applications that is part of the GIRAF system.}
	\item \textit{List on applications that the user is able to use, according to the user's capabilities.}
	\item \textit{List only applications if their usage is not limited by the current location or time.}
	\item \textit{Only display application names if the user is able to read.}
\end{itemize}
\end{quote}

Common to all these, is that they were either difficult to test, or simply not implemented (some may have been implemented to some degree in 2011, but then removed by the 2012 team). The difficulty stems from the necessity of manipulating test data, that is hard coded into the local database of each device. We decided it would be more reasonable to revisit these test cases when GIRAF's database system became ready for use.

The 2012 report by Andersen et al. \cite{launcher2012} contains no explicit specifications, but does contain a list of use cases. We used these as the basis of our tests:
\begin{quote}
\begin{itemize}
	\item \textit{New guardian log in.}
	\item \textit{Configuring an app for a child.}
	\item \textit{Launching an app for a child in guardian mode.}
	\item \textit{Letting a child use an app for a limited time.}
\end{itemize}
\end{quote}
Note that their report also contains an explanation of each use case. We leave these out for brevity.

\subsection{Test Results}
Below is shown a selection of our test cases and their results.
As mentioned in the introduction of this section, we will only discuss the most interesting results, i.e. tests that led to discovery of significant issues.

\subsubsection*{Changing the Colour of an Application Icon}
\paragraph{Specification:} Configuring an app for a child.
\paragraph{Test case:} Use the colour drawer to change the colour of an application, and check if the change is applied to both the application's icon, and its own user interface. (Some applications do not support changing the colour of their user interface.\jesper{Det her skal vi lige være sikre på.})
\paragraph{Result:} The colour of the icon changed immediately, but when starting the application, it still used the pre-change colour. A little experimenting showed that the change is not applied to the application until the user logs into Launcher again.
\paragraph{Resolution:} The issue was resolved, and the change in colour is now reflected in the application immediately.

\subsubsection*{Applications that Crash Launcher}
\paragraph{Specification:} Launching an app for a child in guardian mode.
\paragraph{Test case:} Open an application with a random child profile selected in the profile selector.
\paragraph{Result:} While not directly related to the intent of the use case, while running the test, we attempted to open what turned out to be a non-working version of a GIRAF application. The application suffered an unhandled exception, causing both it and Launcher to crash. This is not satisfactory, as Launcher should always work on top of the device operating system, denying the children access to device settings when using the device without supervision.
\paragraph{Resolution:} The issue was resolved, by making Launcher catch all exceptions thrown by the GIRAF applications it hosts.

\subsubsection*{Use of Device Buttons}
\paragraph{Specfication:} The device buttons (\textit{Back}, \textit{Home}, and \textit{Multitasking}) should have no effect in the \textit{Logo} and \textit{Authentication} activities, as the user should not be able to ``back out'' of Launcher to the device OS.
\paragraph{Test case:} Use the device buttons in different activities and situations.
\paragraph{Result:} In both the \textit{Logo} and \textit{Authentication} activities, applying the \textit{Home} and \textit{Back} buttons made Launcher restart. The \textit{Multitasking} button always activates the multitasking screen of the device OS.
\paragraph{Resolution:} The \textit{Back} button was suppressed. We were however unable to find a similar method for suppressing the other two buttons. This issue was postponed to a later sprint, when it would be easier to determine if it is worth spending the time needed to solve it.







