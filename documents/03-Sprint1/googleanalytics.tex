\subsection{Google Analytics}

Google Analytics is a tool for gathering statistics on the use of mobile applications and websites. 
It is developed to help marketers reach their target groups, by analysing how their applications are used.

Most of the possibilites of Google Analytics are not directly relevant for the \giraf project, however it has a feature which could be of great use.
Google Analytics can automatically record uncaught exceptions that occur on the users' devices.
This is a useful feature for multiple reasons. The users may not report an application crash to the developers, which prevents the developers from addressing the underlying issue. Furthermore, Google Analytics continuously monitors problems with the applications, even during the eight months each year where no students are assigned to the project. When a new team a students resume the work, they immediately start addressing issues that may have shown up during this period.

Google Analytics is relatively easy to set up, having the following requirements:
\begin{itemize}
\item The device must be connected to the Internet.
\item The Google Analytics SDK must be included in compilation of the application.
\item An XML file with Google Analytics settings for the specific application must be created and included with the application.
\item Each activity must have methods for starting and stopping recording of usage statistics. These two methods are implemented in the \lstinline{onCreate()} and \lstinline{onStop()} methods of each activity. An example is shown in \Cref{lst:googleanalystics}.
\end{itemize}
The XML file should include information such as the \textit{tracking ID}, which allows Google's servers to identify which project the information is related to.

\begin{lstlisting}[caption={The code that needs to be added to \lstinline{OnCreate()} and \lstinline{OnStop()} methods.}, label={lst:googleanalystics}]
  @Override
  public void onStart() {
    super.onStart();
    ... // The rest of the onStart() code.
    EasyTracker.getInstance(this).activityStart(this);  // Add this method.
  }

  @Override
  public void onStop() {
    super.onStop();
    ... // The rest of the onStop() code.
    EasyTracker.getInstance(this).activityStop(this);  // Add this method.
  }
\end{lstlisting}

\subsubsection{Settings for Data Sharing}
When setting up the account at Google Analytics, one is asked to decide whether or not to share the collected analysis data from your applications with other applications.
For this project we have decided to deactivate all data sharing with external applications, since the applications in this project may contain person sensitive information which we do not by any means want to share with anyone.

% https://developers.google.com/analytics/devguides/collection/android/v3/
% http://www.google.com/intl/en_uk/analytics/features/mobile.html
%
% Ekstra stuff
% https://developers.google.com/analytics/devguides/collection/android/v3/exceptions