This appendix describes all front end applications in the \giraf suite.

\paragraph{\launcher} provides an interface for accessing the other tablet applications in a controlled environment, easy to use for both guardians and citizens. 
Through \launcher, the guardians should be able to control what applications the citizens should be allowed to use. In the application interface, \launcher is referred to as \giraf.

\paragraph{Sekvens} allows users to build sentences from pictograms, a central activity in the lives of the citizens \giraf is made to service. Many citizens with autism, especially young children, have difficulty formulating sentences in speech. A central feature of \textit{Sekvens} is also the possibility to save often used sequences of pictograms.

\paragraph{Pictooplæser} is similar to \textit{Sekvens}, but is focused on building more ad hoc sentences, which the application is then able to read aloud using either an existing recording, or an online text-to-speech tool. 

\paragraph{Kategoriværktøjet} allows guardians to manage the categories into which the pictograms are organised, including adding and deleting categories and subcategories.

\paragraph{Oasis App} is an administration tool for manipulating the user information in the database. 

\paragraph{Pictosearch} is not used as a standalone application, but provides other applications with a common interface for searching through the device's collection of pictograms.

\paragraph{Pictotegner} allows the user to create their own pictograms with basic graphics tools, such as a free-hand pen tool, a rectangle tool, a circle tool etc. 

\paragraph{Livshistorier} is also similar to \textit{Sekvens}, but is created specifically for saving pictogram sequences that contain instructions for the citizen's everyday life, e.g. instructions on how to use the bathroom.

\paragraph{Ugeplan} allows the users to build weekly activity schedules using pictograms. Autistic citizens thrive best in highly strutured environments with regularly scheduled days.

\paragraph{Tidstager} provides the ability to time an activity, allowing a guardian to let a citizen use an application, e.g. a game, for a set amount of time. When the time runs out, the device locks, forcing the citizen to move on to the next scheduled activity.

\paragraph{Stemmespillet} is a game, where the citizen controls a car by varying the volume of his or her voice. 

\paragraph{Kategorispillet}
is a game, where the citizen unloads pictograms from a train. The pictograms must be unloaded at different stations, where each station accepts a certain category of pictograms.

\paragraph{Web Ugeplan} is a web-version of \textit{Ugeplan}, specifically design for large touchscreens. A few costumers had access to TV-size touchscreen, which they specifically wished to use for their week schedules.

\paragraph{Webadmin} is a web-based administration tool for manipulating the user information in the database.