This appendix contains code fragments, that are too big to be included as part of the report and that can be omitted..

\section {Debug mode for development}\label{appendix:debugmode}
This section describes the implementation of a debug mode, making developers able to skip certain steps in \launcher, such as the animation screen and the authentication screen.\\

When opening \launcher one is presented \mainactivity, which shows a loading animation, while loading data from the remote database\footnote{Please note, that because the remote database synchronisation was enabled so late in the fourth sprint, debug mode has not been tested.}.
\launcher also requires the users to authenticate themselves, before being given access to the \homeactivity.
While these activities hold meaning in the context of the intended users, a significant amount of debugging time is wasted, as the application is often reinstalled and restarted during development. 

To overcome this problem, we decided to implement a debug mode to simplify and quicken the process of working with \launcher.
The debug mode is controlled from the source code \lstinline|MainActivity| by setting the local fields below (thus, it is not possible to control debug mode at runtime):

\begin{itemize}
\item Enable (true) or disable (false) debug mode entirely, overriding other settings:\\
\lstinline|private final boolean DEBUG_MODE = true;|
\item Skip \lstinline|AuthenticationActivity| activity:\\
\lstinline|private final boolean showAuthentication = false;|
\item Skip the animation on \lstinline|MainActivity| activity:\\
\lstinline|private final boolean showMainAnimation = false;|
\item Login either as a guardian or child when skipping authentication:\\
\lstinline|private final boolean loginAsChild = false;|
\end{itemize}

When the above fields are set, debug mode is enabled globally in \launcher from the \lstinline|onCreate()| method in \lstinline|MainActivity| through the call showed in \cref{lst:debugmode:enable}.

\begin{lstlisting}[caption={Enable debug mode from \lstinline|MainActivity|.},label={lst:debugmode:enable}]  
if(DEBUG_MODE)
  LauncherUtility.enableDebugging(DEBUG_MODE, loginAsChild, this);
\end{lstlisting}

The \lstinline|enableDebugging()| method is seen in \cref{lst:debugmode:enablemethod}.
It needs a reference of the calling activity to show debug information in the active activity.

\begin{lstlisting}[caption={Enable debug mode by calling \lstinline|enableDebugging()|.},label={lst:debugmode:enablemethod}]  
public static void enableDebugging(boolean debugging, boolean loginAsChild, Activity activity) {
  DEBUG_MODE = debugging;
  DEBUG_MODE_AS_CHILD = loginAsChild;

  ShowDebugInformation(activity);
}
\end{lstlisting}

The code in \cref{lst:debugmode:show} is used to set the necessary views, as to inform the developer that debug mode is enabled.

\begin{lstlisting}[caption={Show a debug information on activity if debug is enabled.},label={lst:debugmode:show}]  
public static void ShowDebugInformation(Activity a) {
  if (DEBUG_MODE) {
    LinearLayout debug = (LinearLayout) a.findViewById(R.id.debug_mode);
    TextView textView = (TextView) a.findViewById(R.id.debug_mode_text);
    textView.setText(a.getText(R.string.giraf_debug_mode) + " " + (DEBUG_MODE_AS_CHILD ? a.getText(R.string.giraf_debug_as_child) : a.getText(R.string.giraf_debug_as_guardian)));
    debug.setVisibility(View.VISIBLE);
    try {
      Thread.sleep(200);
    } catch (InterruptedException e) {
      e.printStackTrace();
    }
  }
}
\end{lstlisting}


\section{Launching Google Play}

The code in \cref{lst:launchergoogleplay} describes the \lstinline|OnClickListener| for the \textbf{''Butik''} button in the ''Apps'' pane of settings.
It attempts to open the Play Store app - If it is not installed on the device, it opens Play Store in the browser instead.
\begin{lstlisting}[caption={The OnClickListener for the googlePlayButton, launching the Play Store correctly}, label={lst:launchergoogleplay}]
googlePlayButton.setOnClickListener(new View.OnClickListener() {
	public void onClick(View v) {
		// Try to start the Google Play app
		try {
			Intent intent = new Intent();
			intent.setData(Uri.parse(MARKET_SEARCH_APP_URI + PUBLISHER_NAME));
			intent.setFlags(Intent.FLAG_ACTIVITY_NEW_TASK | Intent.FLAG_ACTIVITY_CLEAR_TOP);
			startActivity(intent);
			// If Google Play is not found, parse the url for Google Play website
		} 
		catch (android.content.ActivityNotFoundException e) {
			startActivity(new Intent(Intent.ACTION_VIEW, Uri.parse(MARKET_SEARCH_WEB_URI + PUBLISHER_NAME)));
		}
	}
}
\end{lstlisting}

\section{Derived class of LoadApplicationTask}

\cref{lst:derivedlat} shows the \lstinline|LoadGirafApplicationTask| - a derived class of \lstinline|LoadApplicationTask|.
The methods of the class calls \lstinline|super.fooBar()| to let the super class carry out most of the work, along with initiating the \lstinline|AppUpdater| in that class and specifying which applications to load.

  \begin{lstlisting}[caption={The LoadGirafApplicationTask, derived from LoadApplicationTask. This is the derived class used by GirafFragment to load applications into view. Please note that all comments and the constructor have been removed to make the listing smaller}, label={lst:derivedlat}]
class LoadGirafApplicationTask extends LoadApplicationTask {
	
	@Override
	protected void onPreExecute() {
		if(appsUpdater != null)
		appsUpdater.cancel();
		
		super.onPreExecute();
	}
	
	@Override
	protected HashMap<String, AppInfo> doInBackground(Application... applications) {
		apps = ApplicationControlUtility.getGirafAppsOnDeviceButLauncherAsApplicationList(context);
		applications = apps.toArray(applications);
		appInfos = super.doInBackground(applications);
		
		return appInfos;
	}
	
	@Override
	protected void onPostExecute(HashMap<String, AppInfo> appInfos) {
		super.onPostExecute(appInfos);
		loadedApps = appInfos;
		startObservingApps();
		haveAppsBeenAdded = true;
	}
}
\end{lstlisting}

\section{Marking GIRAF and Android applications}\label{appendix:markingapps}

\cref{lst:addinggirafapplications} contains the code for marking \giraf applications in the ''Apps'' pane of settings, while \cref{lst:addingandroidapplications} contains the code for marking Android applications in the same pane.

\begin{lstlisting}[caption={The methods used for adding or removing a Giraf application to a user}, label={lst:addinggirafapplications}]
ProfileApplicationController pac = new ProfileApplicationController(context);
ProfileApplication pa = new ProfileApplication(currentUser.getId(), app.getApp().getId());
if(pa == null)
	pac.insertProfileApplication(pa);
else
	pac.removeProfileApplicationByProfileAndApplication(app.getApp(), currentUser);
\end{lstlisting}

\begin{lstlisting}[caption={The methods used for adding or removing an Android application to a user. Please note that the documentation has been removed.}, label={lst:addingandroidapplications}]
String activityName = app.getActivity();

if (selectedApps.contains(activityName))
    selectedApps.remove(activityName);
else
    selectedApps.add(activityName);
\end{lstlisting}