This section describes the implementation of a debug mode, making developers able to skip certain steps in \launcher, such as the animation screen and the authentication screen.\\

When opening \launcher one is presented \mainactivity, which shows a loading animation, while loading data from the remote database\footnote{Please note, that because the remote database synchronisation was enabled so late in the fourth sprint, debug mode has not been tested.}.
\launcher also requires the users to authenticate themselves, before being given access to the \homeactivity.
While these activities hold meaning in the context of the intended users, a significant amount of debugging time is wasted, as the application is often reinstalled and restarted during development. 

To overcome this problem, we decided to implement a debug mode to simplify and quicken the process of working with \launcher.
The debug mode is controlled from the source code \lstinline|MainActivity| by setting the local fields below (thus, it is not possible to control debug mode at runtime):

\begin{itemize}
\item Enable (true) or disable (false) debug mode entirely, overriding other settings:\\
\lstinline|private final boolean DEBUG_MODE = true;|
\item Skip \lstinline|AuthenticationActivity| activity:\\
\lstinline|private final boolean showAuthentication = false;|
\item Skip the animation on \lstinline|MainActivity| activity:\\
\lstinline|private final boolean showMainAnimation = false;|
\item Login either as a guardian or child when skipping authentication:\\
\lstinline|private final boolean loginAsChild = false;|
\end{itemize}

When the above fields are set, debug mode is enabled globally in \launcher from the \lstinline|onCreate()| method in \lstinline|MainActivity| through the call showed in \cref{lst:debugmode:enable}.

\begin{lstlisting}[caption={Enable debug mode from \lstinline|MainActivity|.},label={lst:debugmode:enable}]  
if(DEBUG_MODE)
  LauncherUtility.enableDebugging(DEBUG_MODE, loginAsChild, this);
\end{lstlisting}

The \lstinline|enableDebugging()| method is seen in \cref{lst:debugmode:enablemethod}.
It needs a reference of the calling activity to show debug information in the active activity.

\begin{lstlisting}[caption={Enable debug mode by calling \lstinline|enableDebugging()|.},label={lst:debugmode:enablemethod}]  
public static void enableDebugging(boolean debugging, boolean loginAsChild, Activity activity) {
  DEBUG_MODE = debugging;
  DEBUG_MODE_AS_CHILD = loginAsChild;

  ShowDebugInformation(activity);
}
\end{lstlisting}

The code in \cref{lst:debugmode:show} is used to set the necessary views, as to inform the developer that debug mode is enabled.

\begin{lstlisting}[caption={Show a debug information on activity if debug is enabled.},label={lst:debugmode:show}]  
public static void ShowDebugInformation(Activity a) {
  if (DEBUG_MODE) {
    LinearLayout debug = (LinearLayout) a.findViewById(R.id.debug_mode);
    TextView textView = (TextView) a.findViewById(R.id.debug_mode_text);
    textView.setText(a.getText(R.string.giraf_debug_mode) + " " + (DEBUG_MODE_AS_CHILD ? a.getText(R.string.giraf_debug_as_child) : a.getText(R.string.giraf_debug_as_guardian)));
    debug.setVisibility(View.VISIBLE);
    try {
      Thread.sleep(200);
    } catch (InterruptedException e) {
      e.printStackTrace();
    }
  }
}
\end{lstlisting}